% Minimal Runtime System Virtual Assembly Language
%
% File:         minval.tex
% Author:       Bob Walton (walton@acm.org)
% Date:		See \date below.
  
\documentclass[12pt]{article}

\usepackage[T1]{fontenc}
\usepackage{lmodern}
\usepackage{makeidx}
% \usepackage{pictex} (obsolete? not available under CentOS 7)
\usepackage{upquote} % (imported to local directory
                     % ; not available under CentOS 7)
    % Modifies \verb and \verbatim to print ' with
    % the Computer Modern Typewrite font.
    % Also includes the textcomp package.
    % Modifies \verb and \verbatim to print ' with
    % the Computer Modern Typewrite font.
    % Also includes the textcomp package.

\makeindex

\setlength{\oddsidemargin}{0in}
\setlength{\evensidemargin}{0in}
\setlength{\textwidth}{6.5in}
\setlength{\textheight}{8.5in}
\raggedbottom

\setlength{\unitlength}{1in}

% The following attempt to eliminate headers at the bottom of a page.
\widowpenalty=300
\clubpenalty=300
\setlength{\parskip}{3ex plus 2ex minus 2ex}

\pagestyle{headings}
\setlength{\parindent}{0.0in}
\setlength{\parskip}{1ex}

\setcounter{secnumdepth}{5}
\setcounter{tocdepth}{5}
\newcommand{\subsubsubsection}[1]{\paragraph[#1]{#1.}}
\newcommand{\subsubsubsubsection}[1]{\subparagraph[#1]{#1.}}

% Begin \tableofcontents surgery.

\newcount\AtCatcode
\AtCatcode=\catcode`@
\catcode `@=11	% @ is now a letter

\renewcommand{\contentsname}{}
\renewcommand\l@section{\@dottedtocline{1}{0.1em}{1.4em}}
\renewcommand\l@table{\@dottedtocline{1}{0.1em}{1.4em}}
\renewcommand\tableofcontents{%
    \begin{list}{}%
	     {\setlength{\itemsep}{0in}%
	      \setlength{\topsep}{0in}%
	      \setlength{\parsep}{1ex}%
	      \setlength{\labelwidth}{0in}%
	      \setlength{\baselineskip}{1.5ex}%
	      \setlength{\leftmargin}{0.8in}%
	      \setlength{\rightmargin}{0.8in}}%
    \item\@starttoc{toc}%
    \end{list}}
\renewcommand\listoftables{%
    \begin{list}{}%
	     {\setlength{\itemsep}{0in}%
	      \setlength{\topsep}{0in}%
	      \setlength{\parsep}{1ex}%
	      \setlength{\labelwidth}{0in}%
	      \setlength{\baselineskip}{1.5ex}%
	      \setlength{\leftmargin}{1.0in}%
	      \setlength{\rightmargin}{1.0in}%
	      }%
    \item\@starttoc{lot}%
    \end{list}}

\catcode `@=\AtCatcode	% @ is now restored

% End \tableofcontents surgery.

\newcommand{\TILDE}{\textasciitilde}

\newcommand{\CN}[2]%	Change Notice.
    {\hspace*{0in}\marginpar{\sloppy \raggedright \it \footnotesize
     $^{\mbox{#1}}$#2}}
    % Change notice.

\newcommand{\TT}[1]{{\tt \bfseries #1}}
\newcommand{\TTALL}{\tt \bfseries}

\newcommand{\STAR}{{\Large $^\star$}}
\newcommand{\PLUS}[1][]{{$^{+#1}$}}
\newcommand{\QMARK}{{$^{\,\mbox{\footnotesize ?}}$}}
\newcommand{\OPEN}{{$\{$}}
\newcommand{\CLOSE}{{$\}$}}

\newcommand{\key}[1]{{\rm \bfseries #1}}

\newcommand{\itemref}[1]{\ref{#1}$\,^{p\pageref{#1}}$}
\newcommand{\pagref}[1]{p\pageref{#1}}

\newcommand{\EOL}{\penalty \exhyphenpenalty}

\newlength{\figurewidth}
\setlength{\figurewidth}{\textwidth}
\addtolength{\figurewidth}{-0.40in}

\newsavebox{\figurebox}

\newenvironment{boxedfigure}[1][!btp]%
	{\begin{figure*}[#1]
	 \begin{lrbox}{\figurebox}
	 \begin{minipage}{\figurewidth}

	 \vspace*{1ex}}%
	{
	 \vspace*{1ex}

	 \end{minipage}
	 \end{lrbox}

	 \vspace*{-15ex}
	 \centering
	 \fbox{\hspace*{0.1in}\usebox{\figurebox}\hspace*{0.1in}}
	 \end{figure*}}

\newenvironment{indpar}[1][0.3in]%
	{\begin{list}{}%
		     {\setlength{\itemsep}{0in}%
		      \setlength{\topsep}{0in}%
		      \setlength{\parsep}{1ex}%
		      \setlength{\labelwidth}{#1}%
		      \setlength{\leftmargin}{#1}%
		      \addtolength{\leftmargin}{\labelsep}}%
	 \item}%
	{\end{list}}

\newenvironment{itemlist}[1][1.2in]%
	{\begin{list}{}{\setlength{\labelwidth}{#1}%
		        \setlength{\leftmargin}{\labelwidth}%
		        \addtolength{\leftmargin}{+0.2in}%
		        \renewcommand{\makelabel}[1]{##1\hfill}}}%
	{\end{list}}

\begin{document}
        
\begin{center}
\Large \bf
Minimal Runtime System\\
Virtual Assembly Language\\[0.5ex]
\huge \bf
MINVAL
\end{center}
\begin{center}
\large \bf
(Draft 1a)
\\[0.5ex]
Robert L. Walton\\
November 19, 2019

\bigskip
 
Table of Contents
\end{center}

\bigskip

\tableofcontents 

\newpage

\section{Introduction}

This document describes \key{MINVAL}, the Minimal Runtime System Virtual
Assembly Language.

\section{Overview}

A typical MINVAL statement is:
\begin{center}
\tt int X = Y + Z
\end{center}
This allocates a new variable {\tt X} of type {\tt int}
to the stack, and sets its value to that value of the
variable {\tt Y} plus the value of the variable {\tt Z}.
Most variables in the stack have a size equal to the natural
word size of the computer (typically 32 or 64 bits), or
several times that size: {\tt intd} is a two word (double) integer
and {\tt intq} is a four word (quad) integer.

Variables have names, like {\tt X}, {\tt Y}, and {\tt Z},
and locations in memory where their values are stored.

Locations can be complicated.  The statement:
\begin{center}
\tt Y \TILDE{} Gobbledeegook
\end{center}
makes the variable {\tt Y} synonymous with the
global variable {\tt Gobbledeegook} in the current
module.  {\tt Gobbledeegook} will have a location that
is at a offset address in the current module global memory, and
a type, such as `{\tt dp uns8}', which says it is a
direct pointer to a one byte unsigned integer, a.k.a., a character.
Direct pointers are stored in words.

The location of {\tt X} is an address in
the frame of the current function execution.

The statement
\begin{center}
\tt Z \TILDE{} Gobbledeegook[W]
\end{center}
makes {\tt Z} synonymous with {\tt Gobbledeegook}
indexed by the variable {\tt W}.  If {\tt W} is an
integer, the address of {\tt Z} the address
of {\tt Gobbledeegook} plus the value of {\tt W}
times the size of {\tt uns8}.

Now suppose
\begin{center}
\tt W \TILDE{} Gook
\end{center}
where {\tt Gook} is a global variable of type {\tt int16},
a signed 16 bit integer.  Then to execute
\begin{center}
\tt int X = Y[4] + Z
\end{center}
the computer must:
\begin{enumerate}
\item read the value of {\tt Y} which is the location of {\tt Gobbledeegook}
\item add {\tt 4} to this location
\item read the value of {\tt Y[4]} from this location and convert it
      to an {\tt int}
\item read the value of {\tt W} from the location of {\tt Gook} and
      convert it to an {\tt int}
\item add the value of {\tt W} to the location of {\tt Gobbledeegook}
      to get the location of {\tt Z}
\item read the value of {\tt Z} from this location and convert
      it to an {\tt int}
\item store the sum of the converted values of {\tt Y[4]} and {\tt Z} into {\tt X}
\end{enumerate}

Now there is a very important difference between {\tt X} on one
hand, and {\tt W}, {\tt Y} and {\tt Z} on the other.  When we call an
out-of-line function and then return, we can expect the value of
{\tt X} to be the same as before the call, but the values of
{\tt W}, {\tt Y} and {\tt Z} may have changed.  This is because the out-of-line
function and functions it calls can access the locations of {\tt W}, {\tt Y}
and {\tt Z}, but not the location of {\tt X}.

The expressions that compute locations have a limited structure,
so that the operations used consists of reading variables or constants,
adding values, multiplying values by constants, and shifting values
left or right by a constant number of bits.  More specifically, a
location must be one of the following:
\begin{itemize}
\item a constant, as for global variables
\item the value of the current function execution frame pointer
\item the value of a variable, as for pointer variables
\item a constant offset added to another location
\item a variable offset added to another location, where
      the variable offset is obtained by multiplying together
      one or more values read from other variables and optionally
      multiplying that product by a constant
\end{itemize}

The following apply to computing locations:
\begin{itemize}
\item locations and offsets are {\tt intwrd} values that are
      addresses or offsets of
      memory units whose size in bits is a constant that is a
      power of two; e.g.,
      there can be bit addresses, byte addresses, word
      addresses, page addresses, etc.
\item whenever a variable value is read as part of a location
      computation, it will be first expanded to a {\tt intwrd}
      value, then optionally masked by a constant, and lastly
      optionally shifted left or right by a constant
\item whenever a location and an offset are added, they will
      first be left shifted so the result is a location in
      the smaller of the two units of the input location and
      offset
\end{itemize}

MINVAL has a full set of number types:
{\tt int8}, {\tt uns8},
{\tt int16}, {\tt uns16}, {\tt flt16}, \ldots,
{\tt int128}, {\tt uns128}, {\tt flt128}.
The types {\tt int}, {\tt uns}, {\tt flt} are just these
types for the target machine word size, and the types
The types {\tt intd}, {\tt intq}, {\tt unsd}, {\tt unsq} are just integer
types for twice (double) or four times (quad) the target machine word size.
The {\tt bool} type is a single bit interpreted as {\tt true} if
1 and {\tt false} if zero: it is in essence a 1-bit unsigned integer.

There are also two kinds of user defined types: packed numbers
and structures.

An example packed number is:

\begin{indpar}\begin{verbatim}
packed uns32 my_type:
    [0-7]    uns op_code              // Operation
    [0]      bool has_constant        // Format indicator
    [8-31]   int constant             // Constant
    [8-15]   uns src1                 // Source Register
    [16-23]  uns src2                 // Source Register
    [24-31]  uns des                  // Destination Register

. . . . . . . . . . . .

my_type X = ...
uns op = X.op_code
int c = X.constant
Y ~ X.des
\end{verbatim}\end{indpar}

A packed number is a number ({\tt uns32} in the example)
that contains fields.  Each field occupies a sequence of
bits in the number ({\tt op\_code} occupies bits 0 through 7
in the example), and this sequence is interpreted as a
number, {\tt int}, {\tt uns}, or {\tt flt}, or as a {\tt bool},
and is given
a name ({\tt op\_code}, {\tt has\_constant}, etc.~in the
example).

Notice that fields can overlap.

Examples of structures are:

\begin{indpar}\begin{verbatim}
struct my_type:
    pack
    uns8    type                     // Object Type
    flt64   weight                   // Object Weight
    align   64
    label   extension

struct my_type:
    origin  extension
    align
    flt64   height                   // Object height
    flt64   width                    // Object height

struct my_type:
    origin  extension
    align
    flt64   volume                   // Object volume

struct your_type:
    include my_type   // Copy members of my_type
    dp uns8 name      // Indirect pointer to name
                      // character string

. . . . . . . . . . . .

my_type X = ...
X.type = BOX
X.weight = 55
X.height = 1023
X.width = 572

struct your_type Y
Y.type = BEER
Y.weight = 0.45
Y.volume = 48
Y.name = "John Doe"
\end{verbatim}\end{indpar}

A structure is a byte sequence that contains members
(e.g., {\tt type}, {\tt weight}, {\tt height}, etc. in
the example).
Each member has an offset in bytes from the beginning of the
structure (e.g. {\tt type} is offset 0 and because it is
packed, {\tt weight} has offset 1).  A {\tt label} is like
a zero length member that has no value, and is used to
position other members via the {\tt origin} statement.
The {\tt include} statement copies members from another
structure type.

In this case {\tt extension} is aligned on a 64 bit boundary.
A structure type has an
alignment equal to the greatest common multiple of the
alignments of its members.

Structure types can be extended
(as per the example), and members can overlay each other.
A structure has a size in bytes just large enough to
accommodate all its members.

Structure types per say cannot be used as variable values.
Instead, pointers to structure types must be used.  In the
example {\tt my\_type} and {\tt your\_type} are types of
direct pointers, which are {\tt int} values containing
addresses.  Similarly {\tt X} and {\tt Y} are pointers.
Here {\tt X} is allocated by an out-of-line function call
denoted by {\tt \ldots}, while {\tt Y} is placed into the
current function execution frame.  In spite of this last,
{\tt Y} is still a pointer.

MINVAL has multiple pointer types.  The builtin ones are:
\begin{itemize}
\item[{\bf dp}] Direct Pointer.  An {\tt int} value holding
the byte address of the bytes pointed at.
\item[{\bf ip}] Indirect Pointer.  An {\tt int} value holding
a byte address of a {\tt int} value holding the byte address
of the bytes pointed at.
\item[{\bf op}] Offset Pointer (a.k.a., indirect offset pointer).
An {\tt intd} value half of which holds an {\tt int} offset $O$ and
the other half of which holds and {\tt int}
byte address of an {\tt int} base value $B$.  $B+O$ is the byte
address of the bytes pointed at.
\end{itemize}

In the example above, {\tt name} is a direct pointer to an unsigned
byte, allowing expressions such as {\tt Y.name[i]} for an index variable
{\tt i}.

Some pointer types can be converted to other pointer types.
All the builtin pointer values can be converted to direct
pointers, as is done when calling legacy library functions
written in the C programming language.  Direct pointer values
can be converted to offset pointer values by using a base ($B$)
equal to zero and setting the offset ($O$) equal to the
direct address.

The default pointer type of a structure is direct, but this
can be changed, e.g., to indirect.  The MIN runtime system
requires that pointers be indirect or indirect offset.

New pointer types may also be defined by the user.

By default, functions in MINVAL are inline.  For example,

\begin{indpar}\begin{verbatim}
function int r = max ( int x, int y ):
    if x < y:
        r = y
    else:
        r = x

int x = ...
int y = ...
int z = max ( x, y )
\end{verbatim}\end{indpar}

An assignment statement can contain a single function call
or a single operator, but not both, and not more than one.
For example, given the definition of {\tt max} above:

\begin{indpar}\begin{verbatim}
int x = ...
int y = ...
int z = ...
int w = max ( x, max ( y, z ) )    // Illegal
int w1 = max ( y, z )              // OK
int w2 = max ( x, w1 )             // OK
\end{verbatim}\end{indpar}

Although variables in assignment statement operands cannot be
replaced by expressions involving variables, they can be replaced
by expressions involving constants.  Constants are their own
type, {\tt const}, and named constants can be created by assignment
statements.  For example, using the definition of {\tt max} above:

\begin{indpar}\begin{verbatim}
const x = 5
const y = 1e8
const z = x + y
const w = z / x + y
int u = ...
int v = max ( u, z / x + w * y )
\end{verbatim}\end{indpar}

Constant arithemtic is in effect infinite precision with unbounded
numbers.  Constant numbers can be converted to any of the runtime
number types, but if the result will not fit in a runtime integer,
a compiler error occurs.  This happens, for example, if {\tt 1e20}
is converted to an {\tt int32}.

It is possible to define inline functions that execute at assembly
time:

\begin{indpar}\begin{verbatim}
function const r = max ( const x, const y ):
    if x < y:
        r = y
    else:
        r = x

const x = 2e5
const y = 3e6
const z = 9e4
const w = max ( x, max( y, z ) )
\end{verbatim}\end{indpar}

Inline function definitions may make use of type wildcards.
A name that consists of only an initial {\tt T} optionally
followed by decimal digits is a type wildcard, that denotes
an arbitary type.  Thus the example:

\begin{indpar}\begin{verbatim}
function T r = max ( T x, T y ):
    if x < y:
        r = y
    else:
        r = x

const x = 2e5
int y = 27e4;
int z = max ( x, y )
const w = 34e4;
const v = max ( x, w )
\end{verbatim}\end{indpar}

A wildcard type of a result variable gets its value from the
type of the result variable in a function call.  The exception
is {\tt const}, which can be a wildcard type in a function
definition if and only if all type wildcards in the function definition
are assigned the value {\tt const} and doing so makes
all types in the function definition be {\tt const}
(except for types that are not used as types, but are used as
{\tt const} values, as in `{\tt sizeof ( int )}').

Pointer types can be wildcards which must have names consisting
of an initial {\tt P} optionally followed by decimal digits.
An example is:

\begin{indpar}\begin{verbatim}
function int r = strlen ( P uns8 s ):
    r = (int) C::strlen ( (dp) s ) )
\end{verbatim}\end{indpar}

which converts the point of type {\tt P} to a pointer of
type {\tt dp} (direct pointer) and calls the C programming
language subroutine {\tt C::strlen} with the direct pointer.

MINVAL does not necessarily make any use of registers, other
than as temporaries during the execution of an instruction,
and a several global registers that hold the current function
execution frame address and current module data address.
However, registers can be used to cache data that exists in
RAM memory, or that should exist in RAM memory but does not
due to optimization.  We call such registers software
caches.

Software caches of RAM memory are flushed when an out-of-line
function is called if (1) the memory is not in the currently
executing function frame, or (2) the memory is part of a
section of the currently executing function from to which
a pointer (direct, indirect, offset, etc.) has been computed.

In addition to having a type and optionally a pointer type,
variables may have qualifiers.  The only qualifiers are
{\tt ro} and {\tt vo}.

{\tt ro} variables are read-only
can cannot be written.  However, they may be writable by
code located elsewhere, so it cannot be assumed they will
never change.  Therefore the {\tt ro} qualifier does not
affect software cacheing.

{\tt vo} variables are volatile, and may be changed asynchronously
between any two reads of the variable.  {\tt vo} variables are
\underline{not} cached in software registers.






\printindex

\end{document}
