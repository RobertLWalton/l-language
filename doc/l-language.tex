% Layered Languages Low Level Language (L-Language)
%
% File:         l-language.tex
% Author:       Bob Walton (walton@acm.org)
% Date:		See \date below.
  
\documentclass[12pt]{article}

\usepackage[T1]{fontenc}
\usepackage{lmodern}
\usepackage{makeidx}
\usepackage{upquote}
     % (import to local directory for CentOS 8)
     % Available on CentOS 8.

\makeindex

\setlength{\oddsidemargin}{0in}
\setlength{\evensidemargin}{0in}
\setlength{\textwidth}{6.5in}
\setlength{\textheight}{8.5in}
\raggedbottom

\setlength{\unitlength}{1in}

% The following attempt to eliminate headers at the bottom of a page.
\widowpenalty=300
\clubpenalty=300
\setlength{\parskip}{3ex plus 2ex minus 2ex}

\pagestyle{headings}
\setlength{\parindent}{0.0in}
\setlength{\parskip}{1ex}

\setcounter{secnumdepth}{5}
\setcounter{tocdepth}{5}
\newcommand{\subsubsubsection}[1]{\paragraph[#1]{#1.}}
\newcommand{\subsubsubsubsection}[1]{\subparagraph[#1]{#1.}}

% Begin \tableofcontents surgery.

\newcount\AtCatcode
\AtCatcode=\catcode`@
\catcode `@=11	% @ is now a letter

\renewcommand{\contentsname}{}
\renewcommand\l@section{\@dottedtocline{1}{0.1em}{1.4em}}
\renewcommand\l@table{\@dottedtocline{1}{0.1em}{1.4em}}
\renewcommand\tableofcontents{%
    \begin{list}{}%
	     {\setlength{\itemsep}{0in}%
	      \setlength{\topsep}{0in}%
	      \setlength{\parsep}{1ex}%
	      \setlength{\labelwidth}{0in}%
	      \setlength{\baselineskip}{1.5ex}%
	      \setlength{\leftmargin}{0.8in}%
	      \setlength{\rightmargin}{0.8in}}%
    \item\@starttoc{toc}%
    \end{list}}
\renewcommand\listoftables{%
    \begin{list}{}%
	     {\setlength{\itemsep}{0in}%
	      \setlength{\topsep}{0in}%
	      \setlength{\parsep}{1ex}%
	      \setlength{\labelwidth}{0in}%
	      \setlength{\baselineskip}{1.5ex}%
	      \setlength{\leftmargin}{1.0in}%
	      \setlength{\rightmargin}{1.0in}%
	      }%
    \item\@starttoc{lot}%
    \end{list}}

\catcode `@=\AtCatcode	% @ is now restored

% End \tableofcontents surgery.

\newcommand{\TILDE}{\textasciitilde}

\newcommand{\CN}[2]%	Change Notice.
    {\hspace*{0in}\marginpar{\sloppy \raggedright \it \footnotesize
     $^{\mbox{#1}}$#2}}
    % Change notice.

\newcommand{\TT}[1]{{\tt \bfseries #1}}
\newcommand{\TTALL}{\tt \bfseries}

\newcommand{\STAR}{{\Large $^\star$}}
\newcommand{\PLUS}[1][]{{$^{+#1}$}}
\newcommand{\QMARK}{{$^{\,\mbox{\footnotesize ?}}$}}
\newcommand{\OPEN}{{$\{$}}
\newcommand{\CLOSE}{{$\}$}}

\newcommand{\ABV}{-{}-{}->}
\newcommand{\MA}{{\em ma}\QMARK}
\newcommand{\TS}{\hspace*{0in}\tt}

\newcommand{\key}[1]{{\rm \bfseries #1}}
\newcommand{\ttkey}[1]{{\tt \bfseries #1}}
\newcommand{\emkey}[1]{{\em \bfseries #1}}
\newcommand{\skey}[2]{{\rm \bfseries #1#2}}
\newcommand{\ikey}[2]{{\rm \bfseries #1}}
\newcommand{\tttkey}[1]{{\tt \bfseries <#1>}}
\newcommand{\ttakey}[1]{{\tt \bfseries *#1*}}
\newcommand{\ttdkey}[1]{{\tt \bfseries .#1}}

\newcommand{\itemref}[1]{\ref{#1}$\,^{p\pageref{#1}}$}
\newcommand{\pagref}[1]{p\pageref{#1}}
\newcommand{\pagnote}[1]{$\,^{p\pageref{#1}}$}
\newcommand{\stack}[1]{\begin{tabular}[t]{@{}l@{}}#1\end{tabular}}

\newcommand{\EOL}{\penalty \exhyphenpenalty}

\newlength{\figurewidth}
\setlength{\figurewidth}{\textwidth}
\addtolength{\figurewidth}{-0.40in}

\newsavebox{\figurebox}

\newenvironment{boxedfigure}[1][!btp]%
	{\begin{figure*}[#1]
	 \begin{lrbox}{\figurebox}
	 \begin{minipage}{\figurewidth}

	 \vspace*{1ex}}%
	{
	 \vspace*{1ex}

	 \end{minipage}
	 \end{lrbox}

	 \vspace*{-15ex}
	 \centering
	 \fbox{\hspace*{0.1in}\usebox{\figurebox}\hspace*{0.1in}}
	 \end{figure*}}

\newenvironment{indpar}[1][0.3in]%
	{\begin{list}{}%
		     {\setlength{\itemsep}{0in}%
		      \setlength{\topsep}{0in}%
		      \setlength{\parsep}{1ex}%
		      \setlength{\labelwidth}{#1}%
		      \setlength{\leftmargin}{#1}%
		      \addtolength{\leftmargin}{\labelsep}}%
	 \item}%
	{\end{list}}

\newenvironment{itemlist}[1][1.2in]%
	{\begin{list}{}{\setlength{\labelwidth}{#1}%
		        \setlength{\leftmargin}{\labelwidth}%
		        \addtolength{\leftmargin}{+0.2in}%
		        \renewcommand{\makelabel}[1]{##1\hfill}}}%
	{\end{list}}

\begin{document}
        
\begin{center}
\Large \bf
Low Level Layered Language\\[0.5ex]
\huge \bf
L-LANGUAGE
\end{center}
\begin{center}
\large \bf
(Draft 1a)
\\[0.5ex]
Robert L. Walton\\
November 7, 2020

\bigskip
 
Table of Contents
\end{center}

\bigskip

\tableofcontents 

\newpage

\section{Introduction}

This document describes \key{L-Language}, the Layered Language
System Low Level Language.

\section{Overview}

A typical L-Language statement is:
\begin{indpar}\begin{verbatim}
int X = Y - C"0"
\end{verbatim}\end{indpar}
This allocates a new variable {\tt X} of type {\tt int}
and sets its value to the value of the
variable {\tt Y} minus the constant {\tt C"0"} (which is
the character code of the character {\tt 0}).
The `variable' {\tt X} is readable, but after it is
initialized it is not writable.

The following is another example:
\begin{indpar}\begin{verbatim}
av *READ-WRITE* uns8 @bp =@ local[81]
av uns8 @cp = "Hello!"
int i = 0
while ( i < @cp.upper ):
    bp[i] = cp[i]
    next i = i + 1
bp[cp.upper] = 0
\end{verbatim}\end{indpar}

Here `{\tt local[81]}' creats an aligned vector of
81 {\tt uns8} (8-bit unsigned) numbers in the current function
frame and returns an aligned vector pointer, or {\tt av}, to
the vector, marking the vector elements as {\tt *READ-WRITE*}.
{\tt "Hello!"} is a constant vector of {\tt uns8} numbers
and is similar except that it marks the vector elements
{\tt co}, for `constant', which is the implied default qualifier
for {\tt @cp}, and therefore
is not explicitly given.
Vector pointers can be used with indices
to reference elements of their vectors, and have {\tt upper} and
{\tt lower} bounds on these indices.  Here the {\tt lower} bounds
are their defaults, which are {\tt 0}.

Here {\tt @bp} is a variable whose name begins with `{\tt @}' and
whose value is a pointer.  Such a variable has an associated
indirect variable {\tt bp} whose name is missing the initial `{\tt @}'.
The expression {\tt @bp[i]} designates a pointer to the {\tt i}+1'st
element of the vector pointed at by {\tt @bp}, but the expression
{\tt bp[i]} designates the value of the element.  Similarly for
{\tt @cp[i]} and {\tt cp[i]}.\footnote{`{\tt @}' is analogous to
C++ `{\tt \&}' used in a variable declaration, but here `{\tt @}'
can be used with different types of pointers, can be used
without restrictions for structure members, and can be used
with mutable pointers.}

The qualifier {\tt *READ-WRITE*} says that a value can be written,
the default qualifier {\tt co}, or `constant',
says a value will \underline{never} be written no matter what,
the qualifier {\tt ro}, or read-only, says that
the value cannot be written using the variable name given, but might be
written by some other piece of code that accesses the value under another
name, and the qualifier {\tt *VOLATILE*} says a value may be
written by some device other than the processor, or by some process
independent of the current process, at any time.
A {\tt *VOLATILE*} value must also be either {\tt *READ-WRITE*}
or {\tt ro}, with {\tt ro} being the default.

Variables in function frames and module memory
have names, like {\tt X}, {\tt Y}, and {\tt Z}, and values
that are constants.
These values most frequently
have a size equal to the natural
word size of the computer (typically 32 or 64 bits), or
several times that size: {\tt intd} is a two word (double) integer
and {\tt intq} is a four word (quad) integer.
Although the value of a variable is constant, the value may point
at a memory location that is read-write.

An aligned vector pointer is a
quad integer ({\tt intq}) containing:
\begin{itemize}
\item A `base pointer' {\tt int} holding the byte address
of an {\tt int} in memory
that contains the `base (byte) address' of the vector.
Note that the {\tt av} value does \underline{not} contain
the base address, but contains instead this pointer to where
the base address is stored in memory.  This scheme allows
the base address to be change without changing the {\tt av} value.
\item An `offset' {\tt int} that is added to the base address
to form the address of the {\tt uns8} type vector element
which will have index {\tt 0} in the vector.
\item A `lower bound' {\tt int} which is the minimum allowed
value of the index {\tt int}.
\item An `upper bound' {\tt int} which is the maximum allowed
value of the index {\tt int} plus 1.
\end{itemize}

There are other types of pointer.  An {\tt fv}, or `field vector',
is like an {\tt av} aligned
vector except that the offset {\tt int} has a bit address in its
high order part and a field size in bits in its low order part.
The {\tt ap} and {\tt fp} types are
similar but do not have the bounds and cannot be indexed.  Lastly
there is the direct pointer, {\tt dp}, that is just a single {\tt int}
containing the byte address; this is most useful for calling
C language functions.
New pointer types may be defined by the user.

If the value of a variable {\tt @V} is a pointer and the name of
the variable begins with an `{\tt @}', the variable's name with the
initial `{\tt @}' removed, in this case {\tt V}, is called the
indirect variable associated to {\tt @V}
and names the value pointed at by {\tt @V}.
For example:

\begin{indpar}\begin{verbatim}
int X = 5
ap *READ-WRITE* int @Y =@ local
Y = X + 2                // Now Y == 7
Y = Y - 4                // Now Y == 3
X = X + 1                // Illegal!  X is co
ap ro int Z = @Y         // Copies pointer value of Y
                         // Conversion from *READ-WRITE*
                         // to ro is legal.
\end{verbatim}\end{indpar}

Here `{\tt =@ local}' allocates an {\tt int} to the current
function frame, zeros it, and returns an `{\tt ap *READ-WRITE* int}'
pointer to its location.

Instead of making a variable point at a {\tt *READ-WRITE*} location you
can update the constant variable using the {\tt next} construct:
\begin{indpar}\begin{verbatim}
int X = 5
int Y = X + 2           // Now Y == 7; Y is co
next Y = Y - 4          // Now Y == 3; Y is co
Y = Y + 1               // Illegal!  Y is co
\end{verbatim}\end{indpar}
Here `{\tt next Y}' is a new variable, distinct from {\tt Y},
but with the same type, pointer type, qualifiers, and name `{\tt Y}',
which hides the previous variable of the same name.
The advantage of doing this is that it makes compilation more
efficient by keeping variables constant (i.e., {\tt co}), and
it improves debuggability by retaining the different values of
the variable for inspection by a debugger.

L-Language has a full set of number types:
{\tt int8}, {\tt uns8},
{\tt int16}, {\tt uns16}, {\tt flt16}, \ldots,
{\tt int128}, {\tt uns128}, {\tt flt128}.
The types {\tt int}, {\tt uns}, {\tt flt} are just these
types for the target machine word size, and the types
The types {\tt intd}, {\tt intq}, {\tt unsd}, {\tt unsq} are just integer
types for twice (double) or four times (quad) the target machine word size.
The {\tt bool} type is a single bit interpreted as {\tt true} if
1 and {\tt false} if 0: it is in essence a 1-bit unsigned integer.

User defined type values consist of a sequence of bytes containing fields.
Fields in turn can contain subfields.
An example is:

\begin{indpar}\begin{verbatim}
type my type:
             uns32                    // Container for:
    [24-31]  uns8 op code             //   Operation
    [31]     bool has constant        //   Format indicator
    [0-23]   int constant             //   Constant
    [16-23]  uns8 src1                //   Source Register
    [8-15]   uns8 src2                //   Source Register
    [0-7]    uns8 des                 //   Destination Register

. . . . . . . . . . . .

my type X:
    X.op code = 5       // This is an initialization block
    X.src1 = 2          // for X in which X is write-only.
    X.src2 = 3
    X.des = 3
uns op = X.op code      // Now op == 5
int d = X.des           // Now d == 3
ap *READ-WRITE* my type @Y =@ local
Y.op code = 129
fp *READ-WRITE* int @C = @Y.constant
ap *READ-WRITE* int @OP = @Y.op code
next op = OP            // Now op == 129
bool B = Y.has constant // Now B == 1
C = -1234               // Now Y.constant = -1234
\end{verbatim}\end{indpar}

In this example there is one field in a {\tt my type} value,
an unlabeled {\tt uns32} integer.
Inside this unlabeled field there are 6 subfields, the first of which is
an {\tt uns8} integer occupying the highest order 8
bits of the unlabeled field, bits 24-31,
where bits are numbered 0, 1, 2, \ldots{} from
low to high order.  The second subfield is a 1-bit {\tt bool}
value that occupies the high order bit, bit 31, of the unlabelled field.
Note that subfields can overlap.

Defined type values are aligned on byte boundaries when
they are stored in memory.  Therefore the `{\tt op code}' subfield
is on a byte boundary, and
the location of {\tt OP} is an {\tt ap} aligned pointer.  Although
the {\tt constant} subfield is on a byte boundary, it is
shorter than an {\tt uns} integer, and therefore the
location of {\tt C} must be an {\tt fp} field pointer.
If `{\tt op code}' where in bits 23-30 instead of 24-31, it would
not be on a byte boundary and the location of {\tt OP} would
also have to be an {\tt fp} field pointer.

Note that `{\tt Y.op code}' is a {\tt *READ-WRITE*}
integer while `{\tt @Y.op code}' is a {\tt co} pointer to
a {\tt *READ-WRITE*} integer.

Note that names in L-Language can have multiple lexemes, as in
the type name `{\tt my type}', the subfield name `{\tt op code}',
and what L-Language calls the
associated member name `{\tt .op code}' which is used to access
the field.

Another example is:

\begin{indpar}\begin{verbatim}
type my type:
    pack
    uns8    kind             // Object Kind
    [7] bool animal          // True if Animal
    [6] bool vegetable       // True if Vegetable
    flt64   weight           // Object Weight
    align   8
    label   extension

type my type:
    origin  extension
    align
    flt64   height           // Object Height
    flt64   width            // Object Width

type my type:
    origin  extension
    align
    flt64   volume           // Object Volume

type your type:
    include my type   // Copy sub-declarations of my type
    dp uns8 name      // Direct pointer to name
                      // character string

. . . . . . . . . . . .

my type X:
    X.kind = BOX
    X.weight = 55
    X.height = 1023
    X.width = 572

your type Y:
    Y.kind = BEER
    Y.weight = 0.45
    Y.volume = 48
    Y.name = "John Doe's Lager"
\end{verbatim}\end{indpar}

Here the definitions of {\tt my type} and {\tt your type} are each
single statements called declarations.  Each of these two
declarations contains a sequence of sub-declarations, e.g.,
for {\tt my type} the first two sub-declarations are
`{\tt pack}' and `{\tt uns8 kind}'.  There is a current
offset in bytes that starts at {\tt 0} and is updated by each sub-declaration.
A sub-declaration such as `{\tt uns8 kind}' allocates a field
(i.e., {\tt kind})
at the current offset and adds the size of the field to the
current offset.

In the example the fields are {\tt kind}, {\tt weight}, {\tt height}, etc.
Fields can be packed or aligned.  An aligned number has an offset
that is a multiple of the length of the number.
Here fields are initially packed
so that since {\tt kind} has offset 0 bytes and size 1 byte,
{\tt weight} has offset 1 byte.  Subfields {\tt animal}
and {\tt vegetable} are 1-bit values inside {\tt kind}.

The {\tt align 8} sub-declaration moves the current offset
forward to a 8-byte boundary and causes fields beyond it
to be aligned and not packed.  A number is aligned if
its offset is a multiple of its length.  Alignments must be powers of two.
A defined type has an
alignment equal to the least common multiple (in this case just the
largest) of the
alignments of its fields.

A {\tt label} is like a zero length field that has no value and
is used to associate a label with the current offset.
Here {\tt extension} has the offset value of 16 bytes.
The {\tt origin} sub-declaration resets the current offset to the offset
of a given label.
The {\tt include} sub-declaration copies all the sub-declarations
from another user defined type.

Defined types can be extended
(as per the example), and fields can overlay each other.
A defined type value has a size in bytes just large enough to
accommodate all its fields.

Values of {\tt const} type are compile-time values, and are
not available at run-time.  Number constants are infinite
precision, with unbounded integers and rational numbers that
are ratios of unbounded integers.  Number constants
can be converted to run-time numbers during compilation.
However it is an compiler error
if the result will not fit into the runtime number.
This happens, for example, if {\tt 1e20}
is converted to an {\tt int32}.

Quoted strings denote string {\tt const} values that can be
used to populate run-time vectors of unsigned integers.
Quoted strings can also be converted to run-time vectors
with constant elements during compilation.

Lastly there are map {\tt const} values that can hold lists
and dictionaries.  These last can be mutable at compile-time,
but cannot be converted to run-time values.

Expressions, statements, and functions that use only {\tt const} values
execute at compile-time and can be used to compute run-time
constants and other arbitrary code.

By default, functions in L-Language are inline.  For example,

\begin{indpar}\begin{verbatim}
function int r = max ( int x, int y ):
    if x < y:
        r = y
    else:
        r = x

int x = ...
int y = ...
int z = max ( x, y )
\end{verbatim}\end{indpar}

A run-time assignment statement can contain a single function call
or can contain operators on non-{\tt const} data
outside reference expessions, but not both.
For example, given the definition of {\tt max} above:

\begin{indpar}\begin{verbatim}
int x = ...
int y = ...
int z = ...
int u = 5 + max ( x, y )           // Illegal
int w = max ( x, max ( y, z ) )    // Illegal
int w1 = max ( y, z )              // OK
int w2 = max ( x, w1 )             // OK
const c1 = 100
const c2 = 1000
int v = max ( x, c1 + c2 )         // OK
\end{verbatim}\end{indpar}

A run-time statement with operators on non-{\tt const} data
that are outside reference expressions must be a variable
assignment, and all operators must operate on data of the
type of that variable.  For example:

\begin{indpar}\begin{verbatim}
// Assume intq is 4*64 bits and so it can hold a 1e60 number.
int x = 5
int y = 1e8
int d = 3
const e = 1e60
intq z = x + y / d + e
    // All 3 operators operate on intq data
\end{verbatim}\end{indpar}

It is possible to define compile-time functions:

\begin{indpar}\begin{verbatim}
function const r = max ( const x, const y ):
    if x < y:
        r = y
    else:
        r = x

const x = 2e5
const y = 3e6
const z = 9e4
const w = max ( x, max( y, z ) )  // Legal!
\end{verbatim}\end{indpar}

Such functions are not available at run-time, and
are not really inline, as there is no
distinction between inline and out-of-line compile-time functions.

Inline function definitions may make use of type wildcards.
A name that is a single word beginning with {\tt T\$}
is a type wildcard that denotes
an arbitary type.  Thus the example:

\begin{indpar}\begin{verbatim}
function T$r r = max ( T$r x, T$r y ):
    if x < y:
        r = y
    else:
        r = x

const x = 2e5
int y = 27e4
int z = max ( x, y )      // T$r is int, x converts to int.
const w = 34e4
const v = max ( x, w )    // T$r is const, all values are const.
\end{verbatim}\end{indpar}

A wildcard type of a result variable gets its value from the
type of the result variable in a function call.  The exception
is {\tt const}, which can be a wildcard type in a function
definition if and only if all type wildcards in the function definition
are assigned the value {\tt const} and doing so makes
all types in the function definition be {\tt const}
(except for types that are not used as types, but are used as
{\tt const} values, as in `{\tt int..size}').

Pointer types can be wildcards which must have names that are
single words beginning with {\tt P\$}.  A list of qualifiers
can also be a wild card named by a single word beginning with
{\tt Q\$}.  An example is:

\begin{indpar}\begin{verbatim}
function uns r = strlen ( P$s Q$s uns8 s ):
    dp ro uns8 sdp = s
    r = call "strlen" ( sdp )
\end{verbatim}\end{indpar}

which converts the pointer of type {\tt P\$s} to a pointer of
type {\tt dp} (direct pointer) and calls the external C programming
language subroutine {\tt strlen} with the direct pointer.

L-Language does not necessarily make any use of registers, other
than as temporaries during the execution of a single statement,
and a pair of registers that hold the current function
execution frame address and current module data address.
However, registers can be used to cache data that exists in
RAM memory, or that should exist in RAM memory but do not
due to optimization.  We call such registers
\skey{software cach}{es}.

Software caches of a RAM memory location are flushed when an out-of-line
function is called if the location does not have the {\tt co} qualifier.
Attaching the {\tt *VOLATILE*} qualifier to a location
flushes software caches of the location between statements.

A pointer type
has two places where a qualifier may appear, as in

\begin{indpar}\begin{verbatim}
type my control block:
    co ap *VOLATILE* uns32 @cr
    ....
\end{verbatim}\end{indpar}
in which {\tt @cr} is a constant pointing at a volatile
{\tt uns32} location {\tt cr}.

Pointer types cannot be cascaded, but there is a work-around
using defined types:
\begin{indpar}\begin{verbatim}
ap av flt64 p = .....         // Illegal!

struct my pointer
    av flt64 p

. . . . . . . . . .

ap my pointer @q = ...        // OK
flt64 v = q.p[0]              // OK
\end{verbatim}\end{indpar}


\section{Syntax}

In this section we describe the syntax of L-Language programs
and briefly indicate the associated semantics, which is
described in detail in following sections.

\subsection{Lexemes}
\label{LEXEMES}

A L-Language source file is a sequence of bytes that is a UTF-8 encoding
of a sequence of UNICODE characters.  This is scanned into a sequence
of \skey{lexeme}s.

Unless otherwise specified, the term `\key{character}' in this
document means a 32-bit UNICODE character.

Lexemes are defined in terms of
the following character classes:

\begin{indpar}
\emkey{horizontal-space-character}
    \begin{tabular}[t]{rl}
    :::= & characters in UNICODE category \TT{Zs} \\
         & (includes {\em ASCII-single-space}) \\
    $|$  & {\em horizontal-tab-character}
    \end{tabular}
\\
\emkey{vertical-space-character}
    \begin{tabular}[t]{rl}
    :::= & {\em line-feed} $|$ {\em carriage-return} \\
    $|$ & {\em form-feed} $|$ {\em vertical-tab}
    \end{tabular}
\\
\emkey{space-character} :::= {\em horizontal-space-character}
                        $|$ {\em vertical-space-character}
\\[1ex]
\emkey{graphic-character} :::= characters in UNICODE categories
                              \TT{L}, \TT{M}, \TT{N}, \TT{P}, and \TT{S}
\\
\emkey{control-character} :::=
	characters in UNICODE categories \TT{C} and \TT{Z}
\\[1ex]
\emkey{isolated-separating-character} :::= \\
\hspace*{0.5in}
    \begin{tabular}[t]{l}
    characters in UNICODE categories \TT{Ps}, \TT{Pi}, \TT{Pe},
    and \TT{Pf}; \\
    includes \TT{\{ ( [ << >> ] ) \}}
    \end{tabular}
\\
\emkey{separating-character} :::= \TT{|} $|$ {\em isolated-separating-character}
\\[1ex]
\emkey{leading-separator-character} :::=
	\TT{`} $|$ \TT{\textexclamdown} $|$ \TT{\textquestiondown}
\\
\emkey{trailing-separator-character} :::=
	\TT{'} $|$ \TT{!} $|$ \TT{?} $|$ \TT{.} $|$ \TT{:}
	       $|$ \TT{,} $|$ \TT{;}
\\[1ex]
\emkey{quoting-character} :::= \TT{"}
\\[1ex]
\emkey{letter} :::=
    characters in UNICODE category \TT{L}
\\
\emkey{ASCII-digit} :::= \TT{0} $|$ \TT{1} $|$ \TT{2} $|$ \TT{3} $|$ \TT{4}
                     $|$ \TT{5} $|$ \TT{6} $|$ \TT{7} $|$ \TT{8} $|$ \TT{9}
\\
\emkey{digit} :::=
    characters in UNICODE category \TT{Nd}
    (includes {\em ASCII-digits})
\\
\emkey{lexical-item-character} :::=
	\begin{tabular}[t]{l}
        {\em graphic-character} other than \\
	{\em separating-character} or \TT{"}
	\end{tabular}
\end{indpar}

Comments may be placed at the ends of lines:
\begin{indpar}
\emkey{comment}\label{COMMENT} :::=
    \TT{//} {\em comment-character}\,$^\star$
\\[1ex]
\emkey{comment-character} :::= {\em graphic-character}
                          $|$ {\em horizontal-space-character}
\end{indpar}

Lexemes may be separated by {\em white-space}, which
is a sequence of {\em space-characters},
but, with some exceptions mentioned just below, is not itself a lexeme:
\begin{indpar}
\emkey{white-space} :::= {\em space-character}\PLUS{}
\\[0.3ex]
\emkey{horizontal-space} :::= {\em horizontal-space-character}\PLUS{}
\\[0.3ex]
\emkey{vertical-space} :::= {\em vertical-space-character}\PLUS{}
\end{indpar}

The following is a special virtual lexeme:
\begin{indpar}
\emkey{indent}\label{INDENT} ::=
        virtual lexeme inserted just before the first
	{\em graphic} character on a line
\end{indpar}

\ikey{Indent lexemes}{indent lexeme} have no characters, but
do have an \key{indent}, which is the indent of
the graphic character after the indent lexeme.
The \key{indent} of a character is the number
of columns that precede the character in the character's physical line.
A single space takes 1 column, and tabs are set every 8 columns.
Other horizontal space characters (e.g., {\tt <NBSP>}) take up 1 column and also
produce a warning message.
Indent lexemes are used to form logical lines and blocks
(\itemref{LOGICAL-LINES-BLOCKS-AND-STATEMENTS}).

One kind of {\em vertical-space} is given special distinction:
\begin{indpar}
\emkey{line-break}\label{LINE-BREAK} ::=
	\begin{tabular}[t]{l}
        {\em vertical-space} containing exactly one {\em line-feed}
	\end{tabular}
\end{indpar}

This is the {\em line-break} lexeme.

{\em Horizontal-\EOL space-\EOL characters}\label{ILLEGAL-CHARACTERS}
other than single
space are illegal inside {\em quoted-string} lexemes (defined below).
{\em Vertical-space} that has \underline{no} {\em line-feeds} is
illegal (see below).
{\em Control-characters} not in {\em white-space} are illegal.
Characters that have no UNICODE category are {\em unrecognized-characters}
and are illegal:
\begin{indpar}
\emkey{misplaced-horizontal-space-character} :::= \\
\hspace*{0.5in}
    {\em horizontal-space-character}, other than ASCII-single-space
\\[0.3ex]
\emkey{misplaced-vertical-space-character} :::= \\
\hspace*{0.5in}{\em vertical-space-character} other than {\em line-feed}
\\[0.3ex]
\emkey{illegal-control-character} :::= \\
\hspace*{0.5in}
    \begin{tabular}[t]{l}
    {\em control-character},
    but \underline{not} a {\em horizontal-space-character} or \\
    {\em vertical-space-character}
    \end{tabular}
\\[0.3ex]
\emkey{unrecognized-character} :::= \\
\hspace*{0.5in}
    \begin{tabular}[t]{l}
    character with no UNICODE category or \\
    with a category other than
    \TT{L}, \TT{M}, \TT{N}, \TT{P}, \TT{S}, \TT{C}, or \TT{Z}
    \end{tabular}
\end{indpar}

Sequences of these characters generate warning messages,
but are otherwise like {\em horizontal-space}:
\begin{indpar}
\emkey{misplaced-horizontal} :::=
    {\em misplaced-horizontal-space-character}\PLUS{}
\\[0.3ex]
\emkey{misplaced-vertical} :::=
    {\em misplaced-vertical-space-character}\PLUS{}
\\[0.3ex]
\emkey{illegal-control} :::= {\em illegal-control-character}\PLUS{}
\\[0.3ex]
\emkey{unrecognized} :::= {\em unrecognized-character}\PLUS{}
\end{indpar}

{\em Misplaced-horizontal} only exists inside a {\em quoted-string},
but the other three sequences can appear anywhere.  When they occur,
these sequences generate warning messages, but otherwise they behave
like {\em horizontal-space}.  Specifically, outside {\em quoted-strings}
and {\em comments} these sequences can be used to separate other lexemes,
just as {\em horizontal-space} can be used,
whereas inside {\em quoted-strings} and
{\em comments} these sequences do nothing aside from generating
warning messages.

\begin{boxedfigure}[!p]
\begin{indpar}

\emkey{lexeme}
        \begin{tabular}[t]{rl}
	::= & {\em word} $|$ {\em mark} $|$ {\em number} $|$
	      {\em separator} $|$ {\em quoted-string} $|$ {\em indent} \\
	$|$ & {\em line-break} $|$
	      {\em comment} $|$ {\em end-of-file}
	\end{tabular}
\label{LEXEME}
\\[1ex]
\emkey{strict-separator} :::= {\em isolated-separating-character} $|$
                              \TT{|}\PLUS{}
\\[0.5ex]
\emkey{leading-separator} :::=
	\TT{`}\PLUS{} $|$ 
	\TT{\textexclamdown}\PLUS{} $|$ \TT{\textquestiondown}\PLUS{}
\\[0.5ex]
\emkey{trailing-separator} :::= \TT{'}\PLUS{} $|$
				   \TT{!}\PLUS{} $|$
				   \TT{?}\PLUS{} $|$
				   \TT{.}\PLUS{} $|$
				   \TT{:}\PLUS{} $|$
				   \TT{;} $|$
				   \TT{,}
\\[0.5ex]
\emkey{separator}
    ::= {\em strict-separator} 
    $|$ {\em leading-separator}
    $|$ {\em trailing-separator}
\\[1ex]
\emkey{quoted-string}\label{QUOTED-STRING} :::=
    \TT{"} {\em character-representative}\,\STAR{} \TT{"}
\\[0.3ex]
\emkey{character-representative}\label{CHARACTER-REPRESENTATIVE}
	\begin{tabular}[t]{@{}rl@{}}
	::= & {\em graphic-character} other than \TT{"} \\
	$|$ & {\em ASCII-single-space-character} \\
	$|$ & {\em special-character-representative} \\
	\end{tabular}
\\[0.3ex]
\emkey{special-character-representative} :::= \\
\hspace*{0.5in}
    \TT{<} \{ {\em ASCII-upper-case-letter} $|$ {\em ASCII-digit}
            \}\PLUS{} \TT{>}
\\[1ex]
\emkey{lexical-item} :::= {\em lexical-item-character}\PLUS{}
                       not beginning with \TT{//}
\\[0.5ex]
\emkey{lexical-item} :::= {\em leading-separator}\STAR{}
			  {\em middle-lexeme}\QMARK{}
                          {\em trailing-separator}\STAR{}
\\[0.5ex]
\emkey{middle-lexeme} :::= 
	{\em lexical-item}
	\begin{tabular}[t]{@{}l@{}}
	not beginning with a \\
	{\em leading-separator-character} \\
	or ending with a \\
	{\em trailing-separator-character} \\
	\end{tabular}
\\[0.5ex]
\emkey{number}
    ::= {\em middle-lexeme} with a {\em digit} before any {\em letter}
\\[0.5ex]
\emkey{natural-number}\label{NATURAL-NUMBER}
	:::= {\em ASCII-digit}\PLUS{} not beginning with \TT{0} $|$ \TT{0}
\\[0.5ex]
\emkey{word} :::= {\em middle-lexeme} containing a {\em letter} that is not
                  a {\em number}
\\[0.5ex]
\emkey{mark} :::= {\em middle-lexeme} that is not a {\em word} or
		  {\em number}
\\[0.5ex]
{\em indent} ::= see \pagref{INDENT}
\\[0.5ex]
{\em line-break} ::= see \pagref{LINE-BREAK}
\\[0.5ex]
{\em comment} ::= see \pagref{COMMENT}
\\[0.5ex]
{\em end-of-file} ::= see \pagref{END-OF-FILE}

\end{indpar}
\caption{L-Language Program Lexemes}
\label{L-LANGUAGE-PROGRAM-LEXEMES}
\end{boxedfigure}

The lexemes in a L-Language program are specified in
Figure~\itemref{L-LANGUAGE-PROGRAM-LEXEMES}.  This specification assumes there
are no illegal characters in the input; see page \pageref{ILLEGAL-CHARACTERS}
above to account for such characters.

The symbol `\ttkey{:::=}' is used in syntax equations
that define lexemes or parts of lexemes whose syntactic elements are
character sequences that must \underline{not} be separated by {\em white-space}.
The symbol `\ttkey{::=}'
is used in syntax equations that define sequences of lexemes that may
and sometimes must be separated by {\em white-space}.

There is a special \emkey{end-of-file}\label{END-OF-FILE}
lexeme that occurs only at the end of a file.

Files are scanned into sequences of lexemes which are then divided
into logical lines as per \itemref{LOGICAL-LINES-BLOCKS-AND-STATEMENTS}.
After each logical line is formed,
{\em indent}, {\em comment},
{\em line-break}, and {\em end-of-file} lexemes are deleted
from the logical line.

\ikey{Quoted string lexemes}{quoted strings!concatenated}
\label{QUOTED-STRING-CONCATENATION}
separated by the `\TT{\#}' mark
are glued together if they are in the
same logical line.  Thus
\begin{indpar}\begin{verbatim}
"This is a longer sentence" #
    " than we would like."
"And this is a second sentence."
\end{verbatim}\end{indpar}
is equivalent to
\begin{indpar}\begin{verbatim}
"This is a longer sentence than we would like."
"And this is a second sentence."
\end{verbatim}\end{indpar}
This is useful for
breaking long quoted string lexemes across line continuations.
But there is an important case where there is not an exact equivalence
between the glued and unglued versions.  \TT{"<" \# "LF" \# ">"} is
\underline{not} equivalent to \TT{"<LF>"}.  The former is a 4-character
quoted string, the characters being \TT{<}, \TT{L}, \TT{F},
and \TT{>}.  The latter is a 1-character quoted string, the character
being a line feed.

\ikey{Middle lexemes}{middle lexemes!glued}
in the same logical line are glued together if the first
ends with `\TT{\#}' and the second begins with `\TT{\#}'.
Thus
\begin{indpar}\begin{verbatim}
This is a continued-#
    #middle# #-lexeme.
\end{verbatim}\end{indpar}
is equivalent to
\begin{indpar}\begin{verbatim}
This is a continued-middle-lexeme.
\end{verbatim}\end{indpar}
For compatibility, two consequtive `\TT{\#}' marks may be used
to glue together two quoted strings, as in
\begin{indpar}\begin{verbatim}
"This is a continued-"#
    #"quoted"# #"-string".
\end{verbatim}\end{indpar}
which is equivalent to
\begin{indpar}\begin{verbatim}
"This is a continued-quoted-string".
\end{verbatim}\end{indpar}


A \emkey{special-character-representative} can consist of
a UNICODE character name surrounded by angle brackets.  Examples are
\TT{<NUL>}, \TT{<LF>}, \TT{<SP>}, \TT{<NBSP>}.  There are three other cases:
\tttkey{Q} represents the doublequote \TT{"}, \tttkey{NL} (new line)
represents a line feed (same as \TT{<LF>}), and \tttkey{UUC} represents
the `\key{unknown UNICODE character}' which in turn is used to represent
illegal UTF-8 character encodings.

A {\em special-character-representative} can also consist of
a hexadecimal UNICODE character code, which must begin with a digit.
Thus \TT{<0FF>} represents \TT{\"y} whereas \TT{<FF>} represents
a form feed.

The definition of a {\em middle-lexeme} is unusual: it is what is left over
after removing {\em leading-separators} and {\em trailing-separators}
from a {\em lexical-item}.  The lexical scan first scans a
{\em lexical-item}, and then removes
{\em leading-separators} and {\em trailing-separators} from it.
Also {\em trailing-separators} are removed
from the end of a {\em lexical-item} by a right-to-left scan, and not
the usual left-to-right scan which is used for everything else.
Thus the {\em lexical-item}
`\TT{\textquestiondown 4,987?,{},::}' yields the
{\em leading-separator} `\TT{\textquestiondown}',
the {\em middle-lexeme} `\TT{4,987}',
and the four {\em trailing-separators} `\TT{?}',
`\TT{,}' `\TT{,}' and `\TT{::}'.

\subsection{Logical Lines, Blocks, and Statements}
\label{LOGICAL-LINES-BLOCKS-AND-STATEMENTS}

Each non-blank physical line begins with an {\em indent} lexeme
that is followed by a
lexeme that is not an {\em indent}, {\em line-break}, or
{\em end-of-file}.

Lexemes are organized into \skey{logical line}s.  A logical line
begins immediately after an {\em indent} lexeme, and the
\key{indent} of the logical line is the
indent of this {\em indent} lexeme (i.e., the indent of the
first graphic character of the logical line).

A logical line ends with the next {\em indent} lexeme whose indent
is not greater than the indent of the logical line, or with an
{\em end-of-file}.  Thus physical
lines with indent greater than that of the current logical line
are \skey{continuation line}s for that logical line.

A code file is a sequence of `\key{top level}' logical lines that
are required to have indent \TT{0}.

A logical line may end with a \key{block} that is itself a sequence of
logical lines that have indents greater than the indent of the
logical line containing the block.
The block is introduced by a `\TT{:}' at the end
of a physical line, provided the `\TT{:}' is not inside brackets
or quotes
(e.g., not inside \TT{(~)} or \TT{`~'}).
If the first {\em indent} lexeme after the
`\TT{:}' has an indent that is \underline{not} greater than the indent
of the logical line containing the `\TT{:}', the block is empty.
Otherwise the indent of this {\em indent} lexeme becomes the
\key{indent} of the block and the indent of all the
logical lines in the block.  The first logical line of the block
starts immediately after this {\em indent} lexeme.
The block ends just before the first
logical line with lesser indent than the block indent, or the end of file.
More specifically, the last logical line of the block ends with an
{\em indent} whose indent is less than the block indent, or with an
{\em end-of-file}.

Examples are:
\begin{indpar}\begin{verbatim}
this is a top level logical line ending with a block:
    this is the first line of the block
    this is the
         second line of the block
    this is the third line of the block:
        this is the first line of a subblock
        this is the second line
                of the subblock:
            this is the only line of a sub-subblock
        this is the third line of the subblock
    this is the fourth line
            of the block:
        this is the only line of the second subblock
    this is the fifth line of the block
         and it ends with an empty subblock:
this is the second top level
     logical line
\end{verbatim}\end{indpar}

A warning message is output if two indents that are being compared
differ by more than \TT{0} and
less than \TT{2} columns, in order to better detect
indentation mistakes.

{\em Line-break} lexemes are effectively ignored.  A sequence
of {\em line-break} lexemes is followed by an {\em indent}
or {\em end-of-file} which is not ignored.
Blank physical lines are represented by sequences of
more than one {\em line-break} lexeme, and are effectively
ignored.

A logical line that contains {\em comments}, but no
lexemes other than {\em comments}, {\em line-breaks}, {\em indents}
and a possible {\em end-of-file}, is
a `\key{comment line}'.

It is an error to begin non-comment logical lines with
a {\em comment}.
{\em Comments} can be used freely in the middle of or at the
end of any logical line, or at the beginning of a comment line.

It is an error for the first logical line of a file
to have an indent that is greater than \TT{0}, the top level
indent.

It is an error for a block to be in the middle of a logical
line.  This means that the first {\em indent} following the
block must have an indent no greater than that of the logical
line containing the block.

Examples are:
\begin{indpar}\begin{verbatim}
// this is a logical line that is a single comment

// this is a logical line that has two
    // comments

this is a logical line // with a comment
     // and another comment
     with three comments // and a last comment

this is a logical line ending with a block:
     First line of the block
     Second line of the block
// Comment that ends block
// Comment that is in error because
    it begins a logical line that this continues

this is a logical line with a block:
     First line of the block
     Second line of the block
  but the block is in error because it is before
  this continuation of the logical line that contains
  the block

this is a logical line ending with a block:
        First line of the block
        Second line of the block
  // comments that end the block, but are in error,
  // because they continue the logical line
  // containing the block
\end{verbatim}\end{indpar}

After a logical line
has been formed, any {\em indent},
{\em comment}, {\em line-break}, and {\em end-of-file}
lexemes in the logical line
are removed from the logical line.  If the result is
empty, e.g., the logical line is a comment line, it is discarded.
Otherwise the
modified logical line becomes a L-Language `\emkey{statement}'.

Therefore a file is a sequence of top-level statements.

Since a logical line can end with a block that itself consists
of a sequence of logical lines, a statement can end with
a block that itself consists of a sequence of statements.


\subsection{Expressions}

Expressions are built from operators, such as \TT{+} and \TT{*},
and primaries, such as variable names and function calls.

Operators are characterized by fixity, precedence, and format.
The L-Language operators are listed in Table~\itemref{L-LANGUAGE-OPERATORS}.

\begin{boxedfigure}[!p]
\begin{center}
\begin{tabular}{|lllllr|}
\hline
Operator & Meaning & Class & Fixity & Format & Precedence \\
\hline
\ttkey{=} & assignment & & infix & binary & L = 1 \\
$AOP$\ttkey{=} & see text & A & infix & binary & 1 \\
$BOP$\ttkey{=} & see text & B & infix & binary & 1 \\
\hline
\ttkey{,} & separator & & nofix & separating & 2 \\
\hline
\ttkey{else} & selection & L & infix & nomix & 3 \\
\hline
\ttkey{if} & condition & L & infix & binary & 4 \\
\hline
\ttkey{BUT NOT} & logical and not & L & infix & binary & 5 \\
\hline
\ttkey{AND} & logical and & L & infix & nomix & 6 \\
\ttkey{OR} & logical or & L & infix & nomix & 6 \\
\hline
\ttkey{NOT} & logical not & L & nofix & unary & 7 \\
\hline
\ttkey{==} & equal & L & infix & left associative & 8 \\
\ttkey{!=} & not equal & L & infix & left associative & 8 \\
\ttkey{<} & less than & L & infix & left associative & 8 \\
\ttkey{<=} & \stack{less than\\or equal} & L & infix & left associative & 8 \\
\ttkey{>} & greater than & L & infix & left associative & 8 \\
\ttkey{>=} & \stack{greater than\\or equal} & L & infix & left associative & 8 \\
\hline
\ttkey{+} & addition & A & infix & sum & 9 \\
\ttkey{-} & subtraction & A & infix & sum & 9 \\
\ttkey{|} & bitwise or & B & infix & nomix & 9 \\
\ttkey{\&} & bitwise and & B & infix & nomix & 9 \\
\ttkey{\textasciicircum} & \stack{bitwise\\exclusive or} & B & infix & nomix & 9 \\
\ttkey{<{}<} & left shift & B & infix & left associative & 9 \\
\ttkey{>{}>} & right shift & B & infix & left associative & 9 \\
\hline
\ttkey{/} & division & A & infix & binary & 10 \\
\ttkey{\%} & modulo & A & infix & binary & 10 \\
\hline
\ttkey{*} & multiplication & A & infix & nomix & 11 \\
\hline
\ttkey{**} & exponentiation & A & infix & binary & H = 12 \\
\hline
\ttkey{+} & no operation & A & prefix & unary & \\
\ttkey{-} & sign change & A & prefix & unary & \\
\ttkey{!} & \stack{bitwise\\complement} & B & prefix & unary & \\
\hline

\end{tabular}
\end{center}

\caption{L-LANGUAGE OPERATORS}
\label{L-LANGUAGE-OPERATORS}
\end{boxedfigure}

Given this, expressions have the following syntax,
where an {\em P-expression}
is an expression all of whose operators that are outside brackets
have precedence equal to or greater than P:

\begin{indpar}\begin{minipage}{6in}
\emkey{expression}\label{EXPRESSION} ::= {\em L-expression}
\\[0.5ex]
\emkey{P-expression}
    \begin{tabular}[t]{@{}rl}
    ::= & \{ {\em (P+1)-expression} $|$ {\em P-operator)} \}\PLUS{} \\
        & where no two {\em (P+1)-expressions} may be adjacent \\ 
        & and each {\em infix-P-operator} must be surrounded by \\
	& {\em (P+1)-expressions} \\
    \end{tabular}
\\[0.5ex]
\emkey{P-operator} ::= {\em nofix-P-operator} $|$ {\em infix-P-operator}
\\[0.5ex]
\emkey{nofix-P-operator} ::= nofix operator of precedence P
\\[0.5ex]
\emkey{infix-P-operator} ::= infix operator of precedence P
\\[0.5ex]
\emkey{(H+2)-expression}
    \begin{tabular}[t]{@{}rl}
    ::= & {\em primary} \\
    $|$ & {\em prefix-operator} {\em (H+2)-expression} \\
    \end{tabular}
\\[0.5ex]
\emkey{primary} ::= {\em non-operator}\PLUS{}
\\[2.0ex]
\hspace*{3em}\begin{tabular}{l}
where P is any precedence in the range [L,H+1]
\end{tabular}
\end{minipage}\end{indpar}

Generally
a {\em P-expression} consists of a sequence of {\em (P+1)-expressions}
and operators of precedence $P$.

The operators in Figure \itemref{L-LANGUAGE-OPERATORS} have precedences in
the range {\em [L,H]}.
Precedence {\em (H+1)} is reserved for the `error operator' which is a
nofix operator inserted by the parser to `fix up' parsing errors
so parsing can continue.
An {\em (H+2)-expression}, while it can be thought of as a sequence
of zero or more prefix operators followed by a single {\em primary},
is parsed so as to execute the prefix operators from right-to-left, each
with a single operand.

An \key{infix} operator must be surrounded by operands,
but a \key{nofix} operator
may or may not have adjacent operands.  A \key{prefix} operator must be followed
by an operand, but must not be preceded by an operand.

A \key{binary} operator must be in an expression consisting of nothing
but two operands separated by an operator.
A \key{unary} operator must be in an expression consisting of nothing
but one operand preceded by an operator.
A \key{nomix} operator must be in an expression consisting of nothing
but operands and identical operators (e.g., \TT{AND} and \TT{OR}
cannot be outside parenthesized operands in the same expression).
The comparison operators are \key{left associative}, e.g.,
`\TT{x < y < z}' is the same as `\TT{(x < y) < z}'.
The \key{separating} format for the comma operator (`\TT{,}')
inserts special \TT{no-operand} primaries
into the expression so the separating operator becomes an infix
operator.  The \key{sum} format allows \TT{+} and \TT{-} to be mixed
in the same expression, but not other operators of the same
precedence.  In this format, \TT{- x} is rewritten as \TT{+ (- x)}
and then \TT{+} is treated as a left associative binary infix operator.

The \TT{NOT} operator is odd in that it is nofix/unary instead of
prefix/unary.  This is done so it can have lower precedence than
the comparison operators, as prefix operators effectively have highest
precedence.

The classes are
\begin{center}
\begin{tabular}{ll}
\ikey{L}{class} & means `logical' \\
\ikey{A}{class} & means `arithmetic' \\
\ikey{B}{class} & means `bitwise' \\
\end{tabular}
\end{center}

\label{RUN-TIME-EXPRESSION-LIMITS}
A statement that executes at run-time
cannot have operators of different classes outside
reference subexpressions (\pagref{REFERENCE-EXPRESSIONS}) and
subexpressions that evaluate at compile-time to a {\tt const} value
(\itemref{COMPILE-TIME}).
Thus a run-time statement may not mix arithmetic and logical,
arithmetic and bitwise, or bitwise and logical run-time
operators outside reference expressions and expressions with
{\tt const} value.

A statement that executes at run-time cannot have class L, A, or B
operators and a call of an inline function,
except it can have reference expression
calls and/or operators inside a reference expression.

The following elaborates the meaning of some of the operators
that are used in these expressions:

\begin{tabular}{rp{5.0in}}
\tt x \TT{if} y & evaluates to {\tt x} if {\tt y} evaluates to {\tt true},
                  and to {\tt no-operand} otherwise
\\[0.5ex]
\tt x \TT{else} y & evaluates to {\tt x} if that is not {\tt no-operand}
                  and to {\tt y} otherwise; {\tt no-operand} is only
		  allowed as a first operand to {\tt else}
\\[0.5ex]
\tt x \TT{<} y \TT{<} z & evaluates to {\tt (x < y) AND (y < z)} where
                  {\tt y} is evaluated only once, and similarly for
		  all comparison operators
\\[0.5ex]
\tt \TT{!} x & evaluates on signed integers as if they were represented
               in two's complement by binary values of unbounded size,
	       and similarly for other bitwise operators
\\[0.5ex]
\tt x \TT{**} y & requires that {\tt y} be a {\tt const} non-negative integer;
{\tt x ** 0 == 1} for all {\tt x}
\\[0.5ex]
\tt x $AOP$\TT{=} y & means {\tt next x = x $AOP$ y} if {\tt x}
is a {\tt co} frame variable, and {\tt x = x $AOP$ y} otherwise,
where $AOP$ is any class A binary operator of precedence above 1
\\[0.5ex]
\tt x $BOP$\TT{=} y & means {\tt next x = x $BOP$ y} if {\tt x}
is a {\tt co} frame variable, and {\tt x = x $BOP$ y} otherwise,
where $BOP$ is any class B binary operator of precedence above 1
\end{tabular}

\subsection{Primaries}

A \key{primary} is an {\em expression} that has no operators:
\begin{indpar}
\emkey{primary}
    \begin{tabular}[t]{@{}rll}
    ::= & {\em constant}		& [\pagref{CONSTANTS}] \\
    $|$ & {\em reference-expression}    & [\pagref{REFERENCE-EXPRESSIONS}] \\
    $|$ & {\em function-call}		& [\pagref{FUNCTION-CALLS}] \\
    \end{tabular}
\end{indpar}

\subsubsection{Names}
\label{NAMES}

A \key{name} is a sequence of lexemes used to name things like
variables and functions.  Names are building blocks of primaries.

\begin{indpar}
\emkey{name}\label{NAME} ::=
    {\em initial-name-item} {\em continuing-name-item}\STAR{} \\
\emkey{initial-name-item} ::= {\em name-item} other than {\em natural-number} \\
\emkey{continuing-name-item} ::= {\em name-item} not containing `\TT{.}' \\
\emkey{name-item}\label{NAME-ITEM}
    \begin{tabular}[t]{@{}rl}
    ::= & {\em word} containing no `\TT{.}' following a character
                     that is not a `\TT{.}' \\
        & [i.e., `\TT{.}'s can only be at the \underline{beginning}
	   of the {\em word}] \\
        & [see text about splitting words with embedded `\TT{.}'s] \\
    $|$ & {\em natural-number} \\
    $|$ & {\em quoted-mark} not containing `\TT{.}'s \\
    $|$ & {\em quoted-separator} not containing `\TT{.}'s \\
    \end{tabular} \\
\emkey{quoted-mark} :::= \TT{"} {\em mark} \TT{"} \\
\emkey{quoted-separator} :::= \TT{"} {\em separator} \TT{"}
\end{indpar}

{\em Words} containing embedded `\TT{.}'s are split into
{\em name-items} which contain `\TT{.}'s only at their beginning.
Thus
\begin{center}
\TT{bills.wife.1.weight..size}
\end{center}
is split into
\begin{center}
\TT{bills~~~.wife~~~.1~~~.weight~~~~..size}
\end{center}
However, `\TT{.1}' is not a legal {\em name-item} and so cannot
be part of a legal {\em name}.

Name items beginning with more than one `\TT{.}' are reserved
for use by systems and compilers (e.g., \TT{..size} in the example).
Name items that are words containing `\TT{\$}' or that both
begin and end with `\TT{*}' are
similarly reserved.  For example, words of the form `\TT{T\$}\ldots'
are reserved for use as type wildcards.

A name can abbreviate another name, using the statement:
\begin{indpar}
\emkey{abbreviation-statement} ::=
    {\em abbreviated-name} ~ \ABV{} ~ {\em abbreviation-name}
\end{indpar}
For example:
\begin{center}
\tt bool \ABV{} std bool
\end{center}

Note that it is whole names that are abbreviated, and not parts of
names.

A name may begin with a {\em word} that is a {\em module-abbreviation}
that designates a code module: see~\itemref{MODULE-DECLARATIONS}.
For example {\tt std} abbreviates the builtin standard module.

L-Language uses several kinds of names:

\begin{indpar}
\emkey{simple-name} ::= \TT{word} not containing any `\TT{.}'s \\
\emkey{module-abbreviation} ::= {\em simple-name} \\
\emkey{ma} ::= {\em module-abbreviation} \\
\emkey{pointer-type-name}\label{POINTER-TYPE-NAME} ::=
    \MA{} {\em simple-name}
\\[1ex]
\emkey{basic-name}\label{BASIC-NAME} ::=
	    {\em name} not containing a `\TT{.}', {\em quoted-mark}, or
	    {\em quoted-separator} \\
\emkey{type-name}\label{TYPE-NAME} ::=
    \MA{} {\em basic-name} \\
\emkey{variable-name}\label{VARIABLE-NAME} ::=
    \MA{} {\em basic-name} \\
\emkey{statement-label} ::= {\em basic-name}
    \label{STATEMENT-LABEL} \\
\\[1ex]
\emkey{member-name}\label{MEMBER-NAME}
	::= \begin{tabular}[t]{@{}l@{}}
                        {\em name} beginning with a `\TT{.}', \\
			but not containing a {\em quoted-mark} or
			    {\em quoted-separator} \\
			(note: all `\TT{.}'s in a {\em name} must be at
			 the beginning of the {\em name})
			\end{tabular} \\
\emkey{data-label}\label{DATA-LABEL} ::=
    {\em basic-name} $|$ {\em member-name}
\\[1ex]
\emkey{function-term-name} ::= {\em name}
    \label{FUNCTION-TERM-NAME}
\\[1ex]
\emkey{qualifier-name}\label{QUALIFIER-NAME} ::=
    \ttkey{co} $|$ \ttkey{ro} $|$ \ttakey{READ-WRITE} $|$
    \ttakey{WRITE-ONLY} $|$ \ttakey{VOLATILE} $|$ \ttakey{ATOMIC}
\begin{indpar}
{\tt co} abbreviates `constant' meaning `never changes' \\
{\tt ro} abbreviates `read-only' meaning `other code may change'
\end{indpar}
\emkey{operator-name}
    \begin{tabular}[t]{rl}
    ::= & \TT{if} $|$ \TT{else}
                  $|$ \TT{xor} $|$ \TT{AND} $|$ \TT{OR}
		  $|$ \TT{NOT} $|$ \TT{BUT}  \\
    $|$ & \TT{mod} $|$ \TT{div} $|$ \TT{rem}
    \end{tabular}
\begin{indpar}
{\tt xor} abbreviates `exclusive or' \\
{\tt mod} abbreviates `modulo' as per mathematics \\
{\tt div} abbreviates `divided by' as per elementary integer arithmetic \\
{\tt rem} abbreviates `remainder' as per elementary integer arithmetic
\end{indpar}

\emkey{function-keyword}
    \begin{tabular}[t]{rl}
    ::= & \TT{no} $|$ \TT{not} $|$ \TT{function} $|$ \TT{generic} \\
    $|$ & \TT{"="} $|$ \TT{","} $|$ \TT{"("} $|$ \TT{")"} $|$
          \TT{"["} $|$ \TT{"]"}
    \end{tabular}

\emkey{wild-card}\label{WILD-CARD}
    ::= {\em simple-name} beginning with \TT{T\$}, \TT{P\$}, or \TT{Q\$}
\begin{indpar}
Each {\tt T\$} name is assigned a {\em type-name} \\
Each {\tt P\$} name is assigned a {\em pointer-type-name} \\
Each {\tt Q\$} name is assigned a sequence of zero or more
               {\em qualifier-names}
\end{indpar}

where the following rules should be followed, least there be
various confusing syntax or semantic errors:
\begin{enumerate}
\item
A {\em type-name} should not begin with a {\em module-abbreviation},
{\em qualifier-name}, or {\em pointer-type-name}.
\item
\label{VARIABLE-NAME-RULE}
A {\em variable-name} should not begin with a {\em module-abbreviation},
{\em qualifier-name}, {\em pointer-type-name}, or {\em type-name}.
\item
A {\em function-term-name} should not begin with a {\em module-abbreviation}
or contain {\em function-keywords}.
\item
{\em Names} not used as operators should not contain {\em operator-names}.
\end{enumerate}
\end{indpar}

For example,
the parser treats a sequence of
names in certain contexts as having the form:
\begin{center}
\{\MA {\em pointer-type-name}\}\QMARK{}
\{\MA {\em type-name}\}\QMARK{}
\MA {\em variable-name}
\end{center}
where \MA denotes an optional {\em module-abbreviation},
and while scanning this sequence from left to right,
the parser does \underline{not} back up after identifying
one of the components of the sequence.

\subsubsection{Constants}
\label{CONSTANTS}

A \key{constant} is a value of type \ttkey{const} computed at
compile-time.  There are three kinds of constants:

\begin{indpar}
\emkey{constant}
    \begin{tabular}[t]{rl}
    ::= & {\em number-constant} \\
    $|$ & {\em string-constant} \\
    $|$ & {\em map-constant} \\
    \end{tabular} \\
\emkey{string-constant} ::= {\em quoted-string}
\\[0.5ex]
\emkey{constant-expression}\label{CONSTANT-EXPRESSION}
    ::= {\em expression} evaluating at compile-time to a {\tt const} value
\\[0.5ex]
\emkey{expression} ::= see \pagref{EXPRESSION}
\end{indpar}

A \emkey{number-constant} is a {\em number} lexeme or sequence of
lexemes with specific syntax
that is used to denote a number.  Numbers are stored either
as unbounded integers, or as rational numbers whose numerator
and denominator are unbounded integers that have no common divisor
with the denominator being equal to or greater that 2.  The syntax is:

\begin{indpar}
\emkey{number-constant}\label{NUMBER-CONSTANT}
    \begin{tabular}[t]{@{}cl}
    ::= & {\em decimal-constant} \\
    $|$ & {\em binary-constant} \\
    $|$ & {\em hexadecimal-constant} \\
    $|$ & \TT{+Inf}\index{Inf@\TT{+Inf}}
          ~$|$~ \TT{-Inf}\index{Inf@\TT{-Inf}}
          ~$|$~ \ttkey{NaN}
    \end{tabular}
\\[0.5ex]
\emkey{sign} :::= \TT{+} $|$ \TT{-} \\
\emkey{exponent} :::=
	\{ \TT{e} $|$ \TT{E} \} {\em sign}\QMARK{} {\em dit}\PLUS{}
\\[0.5ex]
\emkey{decimal-constant} \begin{tabular}[t]{@{}rl@{}}
                         ::= & {\em decimal-lexeme} \\
			 $|$ & {\em decimal-constant-prefix} ~
			       {\em decimal-quoted-body} ~
			       {\em exponent}\QMARK{}
			 \end{tabular}
\\[0.5ex]
\emkey{decimal-lexeme} :::= {\em sign}\QMARK{} ~ {\em decimal-integer} ~
			    {\em decimal-fraction}\QMARK{} ~
                            {\em exponent}\QMARK{}
\\[0.5ex]
\emkey{decimal-constant-prefix} :::= {\em sign}\QMARK{} ~ \TT{D\#}
\\[0.5ex]
\emkey{decimal-quoted-body} :::= \TT{"} {\em decimal-integer} ~
				 {\em decimal-fraction}\QMARK{} \TT{"}
\\[0.5ex]
\emkey{decimal-integer}
    :::= {\em dit}\PLUS{} 
         \{ \TT{,} {\em dit} {\em dit} {\em dit} \}\STAR{} \\
\emkey{decimal-fraction} :::=
    \TT{.} \{ {\em dit} {\em dit} {\em dit} \TT{,} \}\STAR{}
           {\em dit}\PLUS{} \\
\emkey{dit}\label{DIT}
	:::= \TT{0} $|$ \TT{1} $|$ \TT{2} $|$ \TT{3} $|$ \TT{4}
                    $|$ \TT{5} $|$ \TT{6} $|$ \TT{7} $|$ \TT{8} $|$ \TT{9}
 \\[0.5ex]
\emkey{binary-constant} ::= {\em binary-constant-prefix} ~
                              {\em binary-quoted-body} ~
			      {\em exponent}\QMARK{}
\\[0.5ex]
\emkey{binary-constant-prefix} :::= {\em sign}\QMARK{} ~ \TT{B\#}
\\[0.5ex]
\emkey{binary-quoted-body} :::= \TT{"} {\em binary-integer} ~
				{\em binary-fraction}\QMARK{} \TT{"}
\\[0.5ex]
\emkey{binary-integer}
    :::= {\em bit}\PLUS{}
           \{ \TT{,} {\em bit} {\em bit} {\em bit} {\em bit} \}\STAR{} \\
\emkey{binary-fraction} :::=
    \TT{.} \{ {\em bit} {\em bit} {\em bit} {\em bit} \TT{,} \}\STAR{}
    {\em bit}\PLUS{} \\
\emkey{bit} :::= \TT{0} $|$ \TT{1}
 \\[0.5ex]
\emkey{hexadecimal-constant} ::= \\
\hspace*{0.5in}{\em hexadecimal-constant-prefix} ~
               {\em hexadecimal-quoted-body} ~
	       {\em exponent}\QMARK{}
\\[0.5ex]
\emkey{hexadecimal-constant-prefix} :::= {\em sign}\QMARK{} ~ \TT{X\#}
\\[0.5ex]
\emkey{hexadecimal-quoted-body} :::= \TT{"} {\em hexadecimal-integer} ~
				     {\em hexadecimal-fraction}\QMARK{} \TT{"}
\\[0.5ex]
\emkey{hexadecimal-integer}
    :::= {\em hit}\PLUS{}
           \{ \TT{,} {\em hit} {\em hit} \}\STAR{} \\
\emkey{hexadecimal-fraction} :::=
    \TT{.} \{ {\em hit} {\em hit} \TT{,} \}\STAR{}
    {\em hit}\PLUS{} \\
\emkey{hit} :::= \TT{0} $|$ \TT{1} $|$ \TT{2} $|$ \TT{3} $|$ \TT{4}
	     $|$ \TT{5} $|$ \TT{6} $|$ \TT{7} $|$ \TT{8} $|$ \TT{9}
	     $|$ \TT{a} $|$ \TT{b} $|$ \TT{c} $|$ \TT{d} $|$ \TT{e} $|$ \TT{f}
	     $|$ \TT{A} $|$ \TT{B} $|$ \TT{C} $|$ \TT{D} $|$ \TT{E} $|$ \TT{F}
\end{indpar}

The integer part of decimal constants may have commas
every 3 digits from the end and the fractional part may have
commas every 3 digits from the decimal point.
Similarly for binary integers and fractions with commas every 4 binary
digits,
and with hexa-decimal integers and fractions with commas every 2
hexa-decimal digits.
If there is a decimal point, there \underline{must}
be at least one integer digit and
one fraction digit.

\TT{NaN} denotes a canonical non-signaling NaN such
as that produced by hardware on the target machine.
\TT{+Inf} denotes positive infinity; \TT{-Inf} denotes negative infinity.

A number constant may be converted to a run-time type such as {\tt int32}.
It is a compile error to convert to an integer type that cannot
hold the exact value of the number constant.
Conversion of a number constant
to a run-time floating type, such as {\tt flt64}, is however
never a compile error.  If necessary the converted value is
{\tt +Inf} or {\tt -Inf}.

A \emkey{string-constant} is just a {\em quoted-string} lexeme
that denotes a character string: see
\pagref{QUOTED-STRING} and \pagref{QUOTED-STRING-CONCATENATION}.

String constants can be used to load run-time vectors
with {\tt uns8}, {\tt uns16}, or {\tt uns32} type elements.
UTf-8, UTF-16, or UTF-32 encodings are used according to element
size.

An \emkey{map-constant} has two parts, a list and a dictionary
(either or both can be empty).  Its syntax is:

\begin{indpar}
\emkey{map-constant}
    \begin{tabular}[t]{rl}
    ::= & \TT{\{} \TT{\}} \\
    $|$ & \TT{\{} {\em map-list} \TT{\}} \\
    $|$ & \TT{\{} {\em map-dictionary} \TT{\}} \\
    $|$ & \TT{\{} {\em map-list}\TT{,} {\em map-dictionary} \TT{\}} \\
    \end{tabular}
\\[0.5ex]
\emkey{map-list} ::= {\em constant-expression}
                     \{ \TT{,} {\em constant-expression} \}\STAR{}
\\[0.5ex]
\emkey{map-dictionary} ::= {\em label-value-pair}
                              \{ \TT{,} {\em label-value-pair} \}\STAR{}
\\[0.5ex]
\emkey{label-value-pair} ::=
    {\em map-label} \TT{=>} {\em constant-expression}
\\[0.5ex]
\emkey{map-label}\label{MAP-LABEL} ::= {\em data-label}
\\[0.5ex]
\emkey{data-label} ::= see \pagref{DATA-LABEL}
\end{indpar}

Maps \underline{cannot} be represented at run-time.

{\em Type-names} and {\em pointer-type-names} can be used at
compile-time as if they were variables of type {\tt const}
with map values.  These map values are partly read-only,
with the read-only part including elements with labels like
{\tt size} for the size in bits of run-time values of the type.
Users can add their own elements if these do not conflict
with the names of the read-only elements.  See \pagref{TYPE-MAPS}.

\subsubsection{Reference Expressions}
\label{REFERENCE-EXPRESSIONS}

A \emkey{reference-expression} names either a pointer to a location in memory
or names the value of such a location.

The most basic {\em reference-expression} is the name of a variable
in the current function frame or current module memory.  Other
reference expressions are made by appending vector indices or structure
member names to more basic {\em reference-expressions}.  The syntax
is:

\begin{indpar}
\emkey{reference-expression}
    \begin{tabular}[t]{rl}
    ::= & {\em variable-name} \\
    $|$ & {\em reference-expression} \TT{[} {\em index-list} \TT{]} \\
    $|$ & {\em reference-expression} {\em member-name} \\
    \end{tabular}
\\[0.5ex]
\emkey{index-list} ::= {\em index} \{ \TT{,} {\em index} \}\STAR{}
\\[0.5ex]
\emkey{index}\label{REFERENCE-INDEX}
    ::= {\em sign}\QMARK{} {\em index-term}
        \{ {\em sign} {\em index-term} \}\STAR{}
\\[0.5ex]
\emkey{sign} ::= \TT{+} $|$ \TT{-}
\\[0.5ex]
\emkey{index-term}
    \begin{tabular}[t]{rl}
    ::= & {\em constant-expression} \\
    $|$ & {\em variable-name} \\
    $|$ & {\em constant-expression} \TT{*} {\em variable-name} \\
    \end{tabular}
\\[0.5ex]
\emkey{constant-expression} ::= see \pagref{CONSTANT-EXPRESSION}
\\[0.5ex]
\emkey{variable-name} ::= see \pagref{VARIABLE-NAME}
\\[0.5ex]
\emkey{member-name} ::= see \pagref{MEMBER-NAME}
\end{indpar}

A {\em member-name} of the form `\TT{.}{\em data-label}\,'
may be used to select a field or subfield
of a user defined type\pagnote{TYPE-DECLARATIONS} value or
an element of a {\em map-dictionary}.

An {\em index} may be used to select an element of a vector
or array in a
user defined type value, or the
element of a vector pointed at by a pointer, or an element
of a {\em map}.  When used to select a {\em map-dictionary}
element, the {\em index} must be a string.  Otherwise the
index is a positive or negative integer.  Bounds imposed
by user defined types or stored in a pointer are used to
check that the index is within range.

Within an {\em index-list} the comma (\TT{,}) is treated
as equivalent to \TT{][}, so, for example, {\tt [x,y]}
is equivalent to {\tt [x][y]}.

A run-time (non-{\tt const})
variable whose name begins with `{\tt @}' and which names a pointer
value has an associated \key{indirect variable}\label{INDIRECT-VARIABLE}
whose name is made
by removing the initial `{\tt @}' from the variable name (e.g.,
the indirect variable associated with {\tt @V} is {\tt V}).
The indirect variable is automatically declared when the direct
variable is declared.
A run-time variable that has not been declared as an indirect variable
is called a \key{direct variable}\label{DIRECT-VARIABLE}.
Note that syntactically direct variable are must arbitrary
{\em variable-names}.

A {\em reference-expression} that is just a direct variable name
can only be used to read the value of the variable.
If a {\em reference-expression} selects members or vector elements
and begins with a direct variable name, the direct variable value
must be a pointer, and the reference expression computes a pointer
to the selected member or element.

A {\em reference-expression} that is just an indirect variable name
can be used to read or write (if qualifiers allow) the value pointed
at by the associated direct variable.
If a {\em reference-expression} selects members or vector elements
and begins with an indirect variable name, the value of the vector
element or member is designated, and may be read and written
(if qualifiers allow).

Compile-time maps are represented interally by pointers to where
the map is stored, so that if {\tt X} is a variable equal to,
i.e., pointing at, a map, then {\tt Y = X} copies the pointer
to the map to the variable {\tt Y}.  By default compile-time
maps can be changed, but it is possible to mark dictionary members
of a map
as unchangeable constants.  The following example illustrates
computation with compile-time maps:
\begin{indpar}\begin{verbatim}
const X = {"A", "B"}
X[0] = "C"                   // Now X is {"C", "B"}.
const Y = X                  // Now X and Y are both {"C", "B"}. 
Y[1] = "D"                   // Now X and Y are both {"C", "D"}. 
const Z = { Y, "M" }         // Now Z is {{"C", "D"}, "M"}.
Z[0].W = "N"                 // Now Z is {{"C", "D", W => "N"}, "M"},
                             // X and Y are both {"C", "D", W => "N"}.
\end{verbatim}\end{indpar}


Inline functions may be defined that syntactically and semantically mimic
{\em reference-\EOL expres\-sions}, and calls to such functions
can be used as {\em reference-expressions} (this is an exception
to the rule that a run-time statement can contain at most one function
call).  If the inline function returns a value, a call to the function
can only be used to read a value from an apparent
location.  However if the inline function returns a location whose
value has writable qualifiers, a call to the function can be used
to write the location.  An example of the latter is:
\label{REFERENCE-EXPRESSION-FUNCTION-EXAMPLE}

\begin{indpar}\begin{verbatim}
// In my vector X, a writable vector with flt64 elements is
// located at X.offset from the address of X and allows index
// range from 0 through X.length-1.
//
type my vector:
    uns offset;
    uns length;
    . . . .
function ap *READ_WRITE* flt64 @x = ( ap my vector @v ) [ uns index ]:
    av *READ-WRITE* flt64 @p = *UNCHECKED*
            ( @v, v.offset, 0, v.length )
        // The *UNCHECKED* function is a builtin function
        // that performs a variety of conversions which
        // violate type checking.  Here it takes @v, adds
        // v.offset to its offset, and returns this ap as an
        // av with bounds 0 and v.length.
    @x = @p[index]
\end{verbatim}\end{indpar}

\subsubsection{Function Calls}
\label{FUNCTION-CALLS}

The syntax of function calls is:

\begin{indpar}
\emkey{function-call}\label{FUNCTION-CALL} ::=
    {\em module-abreviation}\QMARK{}
    {\em call-argument-list}\STAR{}
    {\em call-term} {\em call-term}\STAR{}
\\[0.5ex]
\emkey{call-term}
    \begin{tabular}[t]{rl}
    ::= & {\em call-term-name} {\em call-argument-list}\STAR{} \\
    $|$ & \ttkey{no} {\em call-term-name} \\
    $|$ & \ttkey{not} {\em call-term-name} \\
    \end{tabular}
\\[0.5ex]
\emkey{call-argument-list}\label{CALL-ARGUMENT-LIST}
    \begin{tabular}[t]{rl}
    ::= & \TT{(} {\em actual-argument}
          \{ \TT{,} {\em actual-argument} \}\STAR{} \TT{)} \\
    $|$ & \TT{[} {\em actual-argument}
          \{ \TT{,} {\em actual-argument} \}\STAR{} \TT{]} \\
    $|$ & \TT{()} $|$ \TT{[]} \\
    $|$ & {\em unparenthesized-actual-argument} \\
    \end{tabular}
\\[0.5ex]
\emkey{actual-argument} ::= {\em expression}
\\[0.5ex]
\emkey{unparenthesized-actual-argument} ::=
    {\em constant} $|$ {\em reference-expression}
\\[0.5ex]
\emkey{call-term-name}\label{CALL-TERM-NAME} ::=
    \begin{tabular}[t]{@{}l}
    {\em function-term-name} with quotes \underline{optionally} removed from \\
    {\em quoted-marks} and {\em quoted-separators}
    \end{tabular}
\\[0.5ex]
\emkey{function-term-name} ::= see \pagref{FUNCTION-TERM-NAME}
\\[0.5ex]
\emkey{constant} ::= see \pagref{CONSTANTS}
\\[0.5ex]
\emkey{reference-expression} ::= see \pagref{REFERENCE-EXPRESSIONS}
\\[0.5ex]
\emkey{expression} ::= see \pagref{EXPRESSION}
\\[2.0ex]
NOTE: Two {\em unparenthesized-actual-arguments} cannot be consecutive.
\end{indpar}

Thus a {\em function-call} is a sequence of {\em function-term-names}
and {\em call-argument-lists}.

An {\em unparenthesized-actual-argument} is an {\em actual-argument}
{\tt X} that would normally be in parentheses as {\tt (X)} but the
parentheses have been omitted.  Thus given the example function definition
on page \pageref{REFERENCE-EXPRESSION-FUNCTION-EXAMPLE},
{\tt (v)[i]} may be written as {\tt v[i]}.  Only {\tt (~)}'s may be
omitted, and then only if they surround exactly one {\em actual-argument}
that is a {\em constant} or {\em reference-expression}.

{\em Call-terms} of the form `{\tt no x}' and `{\tt not x}' are
equivalent to `{\tt x(false)}'.

{\em Function-calls} are matched to function prototypes.  The
{\em call-term-names} in a match are identical to the
{\em function-term-names} taken from the prototype being matched, except
that quotes (\TT{"}) in a prototype {\em quoted-mark} or
{\em quoted-separator} may (or may not) be omitted in the
{\em function-call}.  The first
step in matching is to scan the {\em function-call} to identify the
{\em call-term-names}.  There is no parser backing up after this is
done: if the results of this initial scan do not lead to a satisfying
match, the entire call-prototype match fails.  Therefore caution
is necessary in omitting parentheses around
{\em unparenthesized-actual-arguments} when these share lexemes with
{\em function-term-names} in function prototypes that might be
matched to the {\em function-call}.

\subsection{Compile-Time}
\label{COMPILE-TIME}

Variables and values that exist during compilation are called
\key{compile-time}, while variables and values that exist during
program execution are called {\em run-time}.  Compile-time
is actually an important syntactic concept, as run-time
{\em expressions} cannot contain operators of different
classes or both operators and function calls: see
\pagref{RUN-TIME-EXPRESSION-LIMITS}.

Syntactically \key{compile-time} is defined as follows:

\begin{itemize}
\item A variable (i.e., {\em variable-name})
is compile-time if and only if it is of type {\tt const}.
(Compile-time variables do \underline{not} exist at run-time.)
\item A {\em constant} is compile-time.
\item A {\em type-name} and a {\em pointer-type-name} are compile-time
and denote maps describing the type or pointer type.
(At run-time a type {\tt T} is declared by the
equivalent of {\tt type @T = \ldots}, where `{\tt type}' is a
defined type that is
used to distinguish types and contain type parameters, and
similarly for pointer types.)
\item An {\em expression} with an operator is
      compile-time if its lowest precedence operator
      and its operands are compile-time.
      All the standard operators are compile-time.
\item A {\em function-call} is compile-time if and only if all its
      arguments are compile-time and the function called is
      compile-time.
\item A user-defined function is compile-time if and only if all its
      result and argument variables are compile-time (i.e., have
      {\tt const} type) and all its
      {\em statements} are compile-time.
\item Builtin functions are compile-time if their arguments and
      results are all of {\tt const} type.
\item A {\em statement} is compile-time if and only if all its
      variables and expressions are compile-time.
\end{itemize}

A \key{run-time} {\em variable}, {\em expression}, or {\em statement}
is a {\em variable}, {\em expression}, or {\em statement} that is
\underline{not} compile-time.

\subsection{Statements}
\label{STATEMENTS}

The following is a complete list of the kinds of statements:
\begin{indpar}
\emkey{statement}\label{STATEMENT}
    \begin{tabular}[t]{@{}rll}
    ::= & {\em assignment-statement}
        & [\pagref{ASSIGNMENT-STATEMENTS}] \\
    $|$ & {\em control-statement}
        & [\pagref{CONTROL-STATEMENT}] \\
    $|$ & {\em allocation-statement}
        & [\pagref{ALLOCATION-STATEMENTS}] \\
    $|$ & {\em declaration}
        & [\pagref{DECLARATIONS}] \\
    $|$ & {\em include-statement}
        & [\pagref{INCLUDE-STATEMENTS}] \\
    \end{tabular}
\end{indpar}

\subsubsection{Assignment Statements}
\label{ASSIGNMENT-STATEMENTS}

{\em Assignment-statements} have a list of variables on the
left side which receive values from a list of expressions or
a block of code on the right side.  The left-side variables
may be omitted if the values are not needed and the right side
is just a {\em function-call}.  In some cases
the right side produces no values, though the right side is never omitted.

The general form of an {\em assignment-statement} is:
\begin{indpar}
\emkey{assignment-statement} ::= \\
\hspace*{0.5in} \{ {\em assignment-left-side} \TT{=} \}\QMARK{}
	      {\em assignment-right-side}
\\[0.5ex]
\emkey{assignment-left-side} ::= \\
\hspace*{0.5in}
    \{ {\em assignment-result} \TT{,} \}\STAR{}
    {\em assignment-result}
\\[0.5ex]
\emkey{assignment-result}
    \begin{tabular}[t]{@{}rll}
    ::= & {\em result-variable-declaration} \\
    $|$ & {\em next-variable-declaration} \\
    $|$ & {\em reference-expression}
    		& [see \pagref{REFERENCE-EXPRESSIONS}] \\
    $|$ & {\em deferred-result}
    		& [see \pagref{DEFERRED-RESULT}] \\
    \end{tabular}
\\[0.5ex]
\emkey{result-variable-declaration}\label{RESULT-VARIABLE-DECLARATION} ::= \\
\hspace*{0.5in}
        \{ {\em pointer-type-name} {\em qualifier-name}\STAR{} \}\QMARK{} \\
\hspace*{1.0in}
        {\em type-name} {\em variable-name}
\\[0.5ex]
\emkey{next-variable-declaration}\label{NEXT-VARIABLE-DECLARATION}
    ::= \ttkey{next} {\em variable-name}
\\[0.5ex]
\emkey{qualifier-name} ::= see \pagref{QUALIFIER-NAME}
\\[0.5ex]
\emkey{pointer-type-name} ::= see \pagref{POINTER-TYPE-NAME}
\\[0.5ex]
\emkey{type-name} ::= see \pagref{TYPE-NAME}
\\[0.5ex]
\emkey{variable-name} ::= see \pagref{VARIABLE-NAME}
\\[0.5ex]
\emkey{assignment-right-side}\label{ASSIGNMENT-RIGHT-SIDE} ::= \\
\hspace*{1in}
    \begin{tabular}[t]{@{}rll}
        & {\em expression-list}
    		& [see \pagref{EXPRESSION-LIST}] \\
    $|$ & {\em function-call}
    		& [see \pagref{FUNCTION-CALLS}] \\
    $|$ & {\em block}
	        & [see kinds of assignment statement below] \\
    \end{tabular}
\end{indpar}

The following are the different kinds of {\em assignment-statements}:
\begin{indpar}
\emkey{assignment-statement}
    \begin{tabular}[t]{@{}rll}
    ::= & {\em expression-assignment-statement}
    	& [\pagref{EXPRESSION-ASSIGNMENT-STATEMENTS}] \\
    $|$ & {\em block-assignment-statement}
        & [\pagref{BLOCK-ASSIGNMENT-STATEMENTS}] \\
    $|$ & {\em conditional-assignment-statement}
        & [\pagref{CONDITIONAL-ASSIGNMENT-STATEMENTS}] \\
    $|$ & {\em loop-assignment-statement}
        & [\pagref{LOOP-ASSIGNMENT-STATEMENTS}] \\
    \end{tabular}
\end{indpar}

Associated with some of the more complex types of
{\em assignment-statement} there are {\em control-statements}
to control the flow of execution within the more complex
{\em assignment-statement}:
\begin{indpar}
\emkey{control-statement}\label{CONTROL-STATEMENT}
    \begin{tabular}[t]{@{}rll}
    ::= & {\em block-control-statement}
        & [\pagref{BLOCK-CONTROL-STATEMENTS}] \\
    $|$ & {\em loop-control-statement}
        & [\pagref{LOOP-CONTROL-STATEMENTS}] \\
    \end{tabular}
\end{indpar}

If a {\em result-variable-declaration} declares a {\em variable-name}
that begins with a `{\tt @}' and has a pointer type,
an indirect variable is implicitly declared at the same time
whose name is the {\em variable-name} with the `{\tt @}' removed.
A {\em variable-name} that is not indirect is called a direct variable.
See \pagref{INDIRECT-VARIABLE}.

Note that a {\em result-variable-declaration} does not allow
qualifiers on anything but the target of a pointer.  The implicit
qualifier of a declared direct variable is {\tt co}, meaning the
the value of the variable once initially set is never changed.
The qualifiers of an indirect variable are those of the target of
its associated pointer.

A {\em next-variable-declaration} for a {\em variable} {\tt v}
must occur in the scope of either a {\em result-variable-declaration}
for {\tt v} or another {\em next-variable-declaration} for {\tt v}.
Furthermore, {\tt v} cannot be an indirect variable name.
The {\em next-variable-declaration} re-declares {\tt v} making a new
variable that hides the previously declared {\tt v}.  The new variable
has the same types and qualifiers as the previous variable named {\tt v}.

\subsubsubsection{Expression Assignment Statements}
\label{EXPRESSION-ASSIGNMENT-STATEMENTS}

The syntax of an {\em expression-assign\-ment-statement} is:
\begin{indpar}
\emkey{expression-assignment-statement} ::=
    {\em assignment-left-side} ~\TT{=}~ {\em expression-list}
\\[0.5ex]
\emkey{expression-list}\label{EXPRESSION-LIST} ::=
	      {\em expression} \{ \TT{,} {\em expression} \}\STAR{}
\\[2.0ex]
where
\begin{itemize}

\item If there is \underline{more than one} right-side {\em expression},
the number of right-side {\em expressions} must equal the number
of left-side {\em assignment-results}.

\item For run-time {\em expression-assignment-statements}, if
there is \underline{more than one} right-side {\em expression}, all these
{\em expressions} must be {\em reference-expressions} or
compile-time {\em expressions} (thus run-time computation cannot
be done on the right-side except for {\em reference-expression}
indices).

\item For run-time {\em expression-assignment-statements} with
a \underline{single} right-side {\em expression}, this {\em expression}
may be a single run-time {\em function-call},
in which case there may any number of {\em assignment-results}
consistent with the called function
(see \pagref{INLINE-CALL-PROTOTYPE-MATCHING}).

\item For run-time {\em expression-assignment-statements} with
a \underline{single} right-side {\em expression} containing
\underline{no} run-time {\em function-calls}, 
there must be a single {\em assignment-result}, and the right-side
{\em expression} may only contain operators that operate on
the data type specified by this {\em assignment-result}.
Furthermore the parenthesis depth of the right-side {\em expression}
must not be greater then 4 (i.e., {\tt ((((\ldots))))} is the
maximum depth allowed).

\end{itemize}
\end{indpar}

For cases not involving a run-time {\em function-call},
all {\em reference-expression} locations are evaluated first,
then any other operators are evaluated,
then the right-side values are stored in the left-side locations.


\subsubsubsection{Block Assignment Statements}
\label{BLOCK-ASSIGNMENT-STATEMENTS}

The syntax of {\em block-assignment-statements} is:

\begin{indpar}
\emkey{block-assignment-statement} ::= \\
\hspace*{0.5in}\begin{tabular}[t]{l}
        \{ {\em assignment-left-side} \TT{=} \}\QMARK{}
	~ \{ \ttkey{do} {\em block-label}\QMARK{} \}\QMARK{}
	\TT{:} \\
	\TT{~~~~}{\em statement}\STAR{} \\
	\TT{~~~~}{\em exit-subblock}\STAR{}
	\end{tabular} \\
\emkey{block-label} ::= {\em statement-label}
\\[0.5ex]
\emkey{exit-subblock} ::=
    \begin{tabular}[t]{l}
    {\em exit-label} \ttkey{exit}\TT{:} \\
    \TT{~~~~}{\em statement}\STAR{} \\
    \end{tabular} \\
\emkey{exit-label} ::= {\em statement-label}
\\[0.5ex]
\emkey{statement-label} ::= see \pagref{STATEMENT-LABEL} \\
\emkey{statement} ::= see \pagref{STATEMENT} \\
\emkey{block-control-statements}\label{BLOCK-CONTROL-STATEMENTS}
	::= {\em goto-exit-statement} \\
{\em go-to-exit-statement} ::= see \pagref{GO-TO-STATEMENT} \\
\\[1ex]
where
\begin{itemize}
\item The {\tt =} and {\tt do} cannot \underline{both} be omitted.
\item The {\em assignment-left-side} may \underline{not} contain
{\em reference-expressions}.
\end{itemize}
\end{indpar}

The result variables of a {\em block-assignment-statement}
are given values by the {\em statements} within the block
of the {\em block-assignment-statement} using {\em deferred-result}
names:
\begin{indpar}
\emkey{deferred-result}\label{DEFERRED-RESULT} ::=
    {\em next-variable-declaration} $|$
    {\em reference-expression} $|$
    {\em deferred-result}
\end{indpar}
A {\em deferred-result} is just a copy of one of the
block's {\em assignment-results} with any {\em qualifiers}
and {\em types} omitted (these are inferred within the block
from the associated {\em assignment-result}).  An example is:
\begin{indpar}\begin{verbatim}
ap *READ-WRITE* int @x =@ local
x = 5
ap *READ-WRITE* int @y, int z =:
    @y = @x
    y = 6
    z = 7
    int k = 8
int w = x      // Now w == 6.
int u = z      // Now u == 7
// k is not visible here
\end{verbatim}\end{indpar}

There are scoping rules that prevent {\tt k} from being visible
outside the block in which it is defined;
see \itemref{SCOPING}.

A {\em go-to-exit-statement} within a block may exit the block or
enter an {\em exit-subblock} of the block:
\begin{indpar}
\emkey{go-to-exit-statement}\label{GO-TO-STATEMENT} ::=
    \{ \TT{if} {\em bool-expression} \}\QMARK{}
    \ttkey{go to} {\em go-to-label} \TT{exit}
\\[0.5ex]
\emkey{go-to-label} ::= {\em block-label} $|$ {\em exit-label}
\\[0.5ex]
{\em bool-expression}\label{BOOL-EXPRESSION} ::=
    \begin{tabular}[t]{@{}l}
    {\em expression} evaluating at run-time to a {\tt bool} \\
    or at compile-time to the {\em const} value {\tt "TRUE"} or {\tt "FALSE"}
    \end{tabular}
\end{indpar}

Unless a {\em go-to-exit-statement} is executed,
a block exits after the last {\em statement} in the block,
and an {\em exit-subblock} exits its containing block after the last
{\em statement} in the {\em exit-subblock}.
A {\em go-to-exit-statement} in an {\em exit-subblock} may only enter
a \underline{subsequent} {\em exit-subblock} or exit its containing
{\em block-assignment-statement}.

{\em Go-to-exit-statements} define various possible execution
paths through a {\em block-assignment-statement}.
A variable declared within the block is only visible to statements that can be
arrived at only by execution paths that include the variable's
declaration.  An execution path cannot include
two {\em result-variable-declarations} of the same variable.

All the execution paths must set each of the
block's {\em assignment-result} variables exactly once, except for
{\em assignment-results} of the form
`{\tt next} {\em variable}', which if not set at all will
automatically be given the value {\em variable} had at the beginning
of the {\em block-assignment-statement}.

Also, if `{\tt next} {\em variable}' is used as an
{\em assignment-result} of some {\em statement} within a
{\em block-assignment-statement} that is not within the scope of
a {\em result-variable-declaration} for the {\em variable}
that is also within the {\em block-assignment-statement},
then `{\tt next} {\em variable}' will be automatically added
to the {\em assignment-left-side} of the {\em block-assignment-statement},
if it is not already there.  For example:
\begin{indpar}\begin{verbatim}
int x = 5
do:
    next x = x + 1
\end{verbatim}\end{indpar}
is equivalent to:
\begin{indpar}\begin{verbatim}
int x = 5
next x = do:
    next x = x + 1
\end{verbatim}\end{indpar}


\subsubsubsection{Conditional Assignment Statements}
\label{CONDITIONAL-ASSIGNMENT-STATEMENTS}

A {\em conditional-assignment-statement}
is syntactic sugar for a {\em block-assignment-statement}
with conditional {\em go-to-exit-statements}
and {\em exit-subblocks} (\itemref{BLOCK-ASSIGNMENT-STATEMENTS}).
{\em Conditional-assignment-statements} have the syntax:

\begin{indpar}
\emkey{conditional-assignment-statement} ::= \\
\hspace*{0.5in}\begin{tabular}[t]{l}
        \{ {\em assignment-left-side} \TT{=} \}\QMARK{} \ttkey{if}\TT{:} \\
	\TT{~~~~}{\em expression-assignment-statement}\STAR{} \\
	\TT{~~~~}{\em bool-expression}\TT{:} \\
	\TT{~~~~~~~~~}{\em statement}\STAR{} \\
	\TT{~~~~}{\em expression-assignment-statement}\STAR{} \\
	\TT{~~~~}{\em bool-expression}\TT{:} \\
	\TT{~~~~~~~~~}{\em statement}\STAR{} \\
	\TT{~~~~}\ldots\ldots\ldots\ldots\ldots\ldots\ldots\ldots\ldots \\
	\TT{~~~~}\ttkey{else}\TT{:} \\
	\TT{~~~~~~~~~}{\em statement}\STAR{} \\
	\TT{~~~~}{\em exit-subblock}\STAR{}
	\end{tabular}
\end{indpar}

An example is:
\begin{indpar}\begin{verbatim}
int x = 5
int y = 6
int z = if:
    int sum = x + y         // Sets sum = 11
    int product = x * y     // Sets product = 30
    sum < product:
        z = sum             // Sets z = 11
    else:
        z = product         // Is NOT executed
// Now z = 11
\end{verbatim}\end{indpar}
in which the {\em conditional-assignment-statement} is equivalent to:
\begin{indpar}\begin{verbatim}
int z =:
    int sum = x + y                 // Sets sum = 11
    int product = x * y             // Sets product = 30
    if sum < product go to E1 exit  // goes to E1 exit
    z = product                     // Is NOT executed
    E1 exit:
        z = sum                     // Sets z = 11
// Now z = 11
\end{verbatim}\end{indpar}

In words, each {\em bool-expression} turns into a
{\em go-to-exit-statement}, and the subblock of {\em statements}
after the {\em bool-expression} become the {\em statements}
of the corresponding {\em exit-subblock}.

\subsubsubsection{Loop Assignment Statements}
\label{LOOP-ASSIGNMENT-STATEMENTS}

A {\em loop-assignment-statement} has the syntax:

\begin{indpar}
\emkey{loop-assignment-statement} ::= \\
\hspace*{0.5in}\begin{tabular}[t]{l}
        \{ {\em iteration-result} \{ \TT{,} {\em iteration-result} \}
	   \TT{=} \}\QMARK{} {\em iteration-control}\TT{:} \\
	\TT{~~~~}{\em statement}\STAR{} \\
	\end{tabular}
\\[0.5ex]
\emkey{iteration-result}\label{ITERATION-VARIABLE} ::= 
    \TT{next} {\em iteration-variable}
\\[0.5ex]
\emkey{iteration-control} \begin{tabular}[t]{rl}
	             ::= & {\em loop-label}\QMARK{} \ttkey{loop}
		           \{ \{ \ttkey{at most} \}\QMARK{}
			   {\em int-expression} \ttkey{times} \}\QMARK{} \\
		     $|$ & \ttkey{while} {\em bool-expression} \\
		     $|$ & \ttkey{until} {\em bool-expression} \\
		     \end{tabular}
\\[0.5ex]
\emkey{loop-label} ::= {\em statement-label}
\\[0.5ex]
\emkey{statement-label} ::= see \pagref{STATEMENT-LABEL} \\
\\[0.5ex]
\emkey{int-expression} ::= {\em expression} evaluating to an {\tt int}
\\[0.5ex]
\emkey{bool-expression} ::= see \pagref{BOOL-EXPRESSION}
\\[0.5ex]
\emkey{loop-control-statement}\label{LOOP-CONTROL-STATEMENTS} ::=
    {\em break-statement} $|$ {\em continue-statement}
\\[0.5ex]
\emkey{break-statement}\label{BREAK-STATEMENT} ::=
    \{ \TT{if} {\em bool-expression} \}\QMARK{} \ttkey{break}
    {\em loop-label}\QMARK{}
\\[0.5ex]
\emkey{contine-statement}\label{CONTINUE-STATEMENT} ::=
    \{ \TT{if} {\em bool-expression} \}\QMARK{} \ttkey{continue}
    {\em loop-label}\QMARK{}
\end{indpar}

A {\em loop-assignment-statement} is the semantic equivalent of
a sequence of zero or more copies of the statement with
its {\em iteration-control} replaced by `{\tt do}', making these copies into
{\em block-assignment-statements}.  Each copy is called an
\key{iteration} of the {\em loop-assignment-statement}.
The number of iterations is
determined at run-time according by the {\em iteration-control}
and {\em loop-control-statements}.

A simple example is:
\begin{indpar}\begin{verbatim}
int sum = 0
int i = 1;
next sum, next i = while sum < 4:
    next sum = sum + i
    next i = i + 1
\end{verbatim}\end{indpar}
which is semantically equivalent to:
\begin{indpar}\begin{verbatim}
int sum = 0
int i = 1;
next sum, next i = do:
    next sum = sum + i
    next i = i + 1
next sum, next i = do:
    next sum = sum + i
    next i = i + 1
next sum, next i = do:
    next sum = sum + i
    next i = i + 1
// Now sum == 6 and i == 4
\end{verbatim}\end{indpar}

However at run-time the variable values of all but the
last 4 iterations of the {\em loop-assignment-statement}
are discarded, which would not be the case if the compiler
actually inserted iterations in the source code. 
This only affects debugging.

The {\em break-statement} exits the current iteration of the
{\em loop-assignment-statement} and prevents further iterations.
A {\em continue-statement} exits the current iteration of the
{\em loop-assignment-statement} but lets the
{\em iteration-control} determine whether there will be any
more iterations.  If there are nested loops, a {\em loop-label}
may be used with these statements to designate which nested iteration
is being exited.

{\em Assignment-results} of the form `{\tt next} {\em variable}'
within the loop {\em statements} will be automatically added
to the list of {\em iteration-results} according to the rules
for doing so given at the end of \itemref{BLOCK-ASSIGNMENT-STATEMENTS}.
Thus the above example could be written as:
\begin{indpar}\begin{verbatim}
int sum = 0
int i = 1;
next sum = while sum < 4
    next sum = sum + i
    next i = i + 1
\end{verbatim}\end{indpar}
or
\begin{indpar}\begin{verbatim}
int sum = 0
int i = 1;
while sum < 4
    next sum = sum + i
    next i = i + 1
\end{verbatim}\end{indpar}

Note that an {\em iteration-result} is just a
{\em next-variable-declaration} used in a specific syntactic
context.  As a consequence, {\em iteration-variables} cannot
be indirect variables (see \pagref{INDIRECT-VARIABLE}).

\subsubsection{Declarations}
\label{DECLARATIONS}

The following is a complete list of the kinds of declarations:
\begin{indpar}
\emkey{declaration}\label{DECLARATION}
    \begin{tabular}[t]{@{}rll}
    ::= & {\em type-declaration}
        & [\pagref{TYPE-DECLARATIONS}] \\
    $|$ & {\em pointer-type-declaration}
        & [\pagref{POINTER-TYPE-DECLARATIONS}] \\
    $|$ & {\em inline-function-declaration}
        & [\pagref{INLINE-FUNCTION-DECLARATIONS}] \\
    $|$ & {\em out-of-line-function-declaration}
        & [\pagref{OUT-OF-LINE-FUNCTION-DECLARATIONS}] \\
    \end{tabular}
\end{indpar}

\subsubsubsection{Type Declarations}
\label{TYPE-DECLARATIONS}

The syntax of a type declaration is:

\begin{indpar}
\emkey{type-declaration}\label{TYPE-DECLARATION}
    ::= \begin{tabular}[t]{l}
        \ttkey{type} {\em defined-type-name} \TT{:} \\
	\TT{~~~~~}{\em type-subdeclaration}\STAR{}
	\end{tabular} \\
\emkey{defined-type-name} ::= {\em type-name} \\
\emkey{type-name} ::= see \pagref{TYPE-NAME}
\\[2ex]
\emkey{type-subdeclaration}
    \begin{tabular}[t]{@{}rl}
    ::= &  {\em field-declaration} \\
    $|$ &  \ttkey{align} {\em alignment}\QMARK{} \\
    $|$ &  \ttkey{pack} \\
    $|$ &  \ttkey{label} {\em field-label} \\
    $|$ &  \ttkey{origin} {\em field-label}
           \{ \{\TT{+}$|$\TT{-}\} {\em offset} \}\QMARK{} \\
    $|$ &  \ttkey{origin} \{\TT{+}$|$\TT{-}\} {\em offset} \\
    $|$ &  \ttkey{include} {\em defined-type-name} \\
    \end{tabular}
\\[2ex]
\emkey{field-declaration}
    \begin{tabular}[t]{@{}rl}
    ::= &  {\em field-without-subfields-declaration} \\
    $|$ &  {\em field-with-subfields-declaration} \\
    \end{tabular}
\\[2ex]
\emkey{field-without-subfields-declaration} ::= \\
\hspace*{0.5in}
        \{ {\em qualifier-name}\STAR{} {\em pointer-type-name} \}\QMARK{} \\
\hspace*{1.0in}
        {\em qualifier-name}\STAR{} {\em field-type-name}~
        {\em field-label}~ {\em field-dimension}\QMARK{}
\\[2ex]
\emkey{field-with-subfields-declaration} ::= \\
\hspace*{0.5in}
    {\em qualifier-name}\STAR{}
    \TT{std}\QMARK{} {\em number-type-name}~ {\em field-label}\QMARK{}
                {\em field-dimension}\QMARK{} \\
\hspace*{0.5in}
    {\em subfield-declaration}\PLUS{}
\\[2ex]
\emkey{field-type-name} ::= \TT{std}\QMARK{} {\em number-type-name}
                        $|$ {\em defined-type-name} \\
\emkey{number-type-name}
    \begin{tabular}[t]{@{}rl}
    ::= &  \ttkey{int} $|$ \ttkey{int8} $|$ \ttkey{int16} $|$ \ttkey{int32}
                       $|$ \ttkey{int64} $|$ \ttkey{int128} \\
    $|$ &  \ttkey{uns} $|$ \ttkey{uns8} $|$ \ttkey{uns16} $|$ \ttkey{uns32}
                       $|$ \ttkey{uns64} $|$ \ttkey{uns128} \\
    $|$ &  \ttkey{flt} $|$ \ttkey{flt16} $|$ \ttkey{flt32} $|$ \ttkey{flt64}
                         $|$ \ttkey{flt128} \\
    \end{tabular}
\\[2ex]
\emkey{field-label}\label{FIELD-LABEL} ::=  {\em data-label} \\
\emkey{data-label} ::=  see \pagref{DATA-LABEL}
\\[2ex]
\emkey{field-dimension} ::=  \TT{[} {\em dimension-size} \TT{]} \\
\emkey{dimension-size} ::=  compile-time {\em expression}
			    with non-negative integer value
\\[2ex]
\emkey{subfield-declaration}
    ::= {\em bit-range} {\em subfield-type-name} {\em subfield-label} \\
\emkey{subfield-type-name}\label{SUBFIELD-TYPE-NAME} ::=
    {\em number-type-name} $|$ \TT{bool} \\
\emkey{subfield-label}\label{SUBFIELD-LABEL} ::=  {\em data-label}
\\[2ex]
\emkey{bit-range}
    \begin{tabular}[t]{@{}rl}
    ::= &  \TT{[} {\em onlybit} \TT{]} \\
    $|$ &  \TT{[} {\em lowhighbits} \TT{]} \\
    $|$ &  \TT{[} {\em lowbit} \TT{-} {\em highbit} \TT{]}
    \end{tabular} \\
\emkey{onlybit} ::= compile-time {\em expression}
		    with non-negative integer value \\
\emkey{lowhighbits} :::= {\em dit}+ \TT{-} {\em dit}+
           ~~~~~ [this is a single lexeme] \\
\emkey{lowbit} ::= compile-time {\em expression}
		   with non-negative integer value \\
\emkey{highbit} ::= compile-time {\em expression}
		    with non-negative integer value \\
\emkey{dit} ::= see \pagref{DIT}
\\[2ex]
\emkey{alignment} ::= compile-time {\em expression}
		      with power of 2 integer value \\
\emkey{offset} ::= compile-time {\em expression}

\begin{itemize}
\item All the compile-time {\em expressions} must evaluate to an integer.
\end{itemize}
\end{indpar}

The {\em type-subdeclarations} are processed in order.  At the
beginning of each, there is a align/pack switch value and an
offset-in-bits integer that determine the offset, relative to the
beginning of each datum of the new data type being declared, of the next
{\em field-declaration} field encountered.  These offsets are in bits.
Initially the align/pack switch is set to align and the offset is
set to zero.

If the align/pack switch is in the \key{align} position and the
next {\em type-subdeclaration} is a {\em field-declaration}, the
current offset will be incremented before becoming the offset
of the field being declared.  The increment will be just enough
to make the offset an exact multiple of the field's type's alignment.
The alignment of a number type is its size in bits.  The alignment
of a defined type is the least common multiple of the alignment of
any of its fields.

An `\ttkey{align} $N$' sub-declaration behaves like an unnamed
field of alignment $N$ and zero length,
and in addition sets the align/pack switch to `align'.

A \ttkey{pack} sub-declaration sets the align/pack switch to `pack'.

An \ttkey{origin} sub-declaration changes the current offset to that
of the given {\em field-label} plus or minus the value of the
given {\em offset}.  If the {\em field-label} is omitted, the
the offset is set to the given {\em offset}.

A \ttkey{label} sub-declaration behaves like the definition of a
zero length field, and just assigns the current offset to the
given {\em field-label}.

A \ttkey{include} sub-declaration copies all the {\em type-subdeclarations}
of the given defined type into the current sequence of
{\em type-subdeclarations}.

If there are multiple {\em type-declarations} with the same
{\em type-name}, all but the first act to append {\em type-subdeclarations}
to the current list of such for the {\em type-name}.

Subfields are parts of the previously declared numeric field.
The bits occupied
by a subfield are given by its {\em bit-range}, where bits are numbered
0, 1, \ldots{} from the low order end of numbers.

A subfield value may have fewer bits than the number-type of the subfield.
For integer types, the value is the low order bits of the integer, with
the high order bits added when the value is read by with adding 0 bits
for unsigned integers or copies of the highest order bit for signed integers.
For floating types, the value is missing low order mantissa bits, which
are added as zeros.  If a value outside the representable range is stored,
it is not an error.  Integer values are truncated, and floating values
have low order mantissa bits dropped (there is no rounding).  However, it is
a compile error to have a floating type whose exponent part plus
1 mantissa bit cannot be stored in the subfield value.

{\em Subfield-labels} and {\em field-labels} have the same standing within
{\em reference-expressions}.
Both have associated {\em member-names} made by adding a single
`\TT{.}' to the beginning of the {\em field-label} or {\em subfield-label}.
For example:

\begin{indpar}\begin{verbatim}
type my type:
    uns8    kind         // Object Kind
    [0] bool animal      // True if Animal
    [1] bool vegetable   // True if Vegetable
    flt weight

my type X:
    // Within this block X is write-only.
    X.kind = HIPPOPOTAMUS
        // Also sets animal and vegetable bits.
    X.weight = 152.34
uns8 kind = X.kind
bool animal = X.animal
bool vegetable = X.vegetable
flt weight = X.weight
\end{verbatim}\end{indpar}

If a {\em field-with-subfields-declaration} contains a
{\em field-dimension} the subfields can be indexed
as an alternative to indexed the field.  For example:

\begin{indpar}\begin{verbatim}
type character attributes:
    uns8 [128]
    [0] bool is graphic

character attributes X:
    int i = 0;
    while i < 128:
        X.is graphic[i] = 32 < i && i < 127
        next i = i + 1
bool line feed is graphic = X.is graphic [C "\n"]
bool A is graphic = X.is graphic [C "A"]
\end{verbatim}\end{indpar}

A {\em field-label} may be direct with associated indirect
{\em field-label} if the field has a pointer type.
However, if the {\em field-label} is a {\em member-name}
beginning with `{\tt .}'s, the `{\tt @}' is placed immediately
after any initial `{\tt .}'s. 
For example:

\begin{indpar}\begin{verbatim}
type my type:
    ap *READ-WRITE* int @x;
    ap *READ-WRITE* flt .@y;

my type X:
    X.@x =@ local
    X..@y =@ local
X.x = 5
X..y = 6.6
\end{verbatim}\end{indpar}

If a {\em defined-type-name} appears in several {\em type-declarations},
the subsequent declarations extend the previous ones: that is, their
lists of {\em type-subdeclarations} are concatenated.

Some uses of a {\em defined-type-name} do not require it to have
been previously declared, or do not require all its extensions to have
been previously declared.  These uses include the builtin {\tt local}
and {\tt global} allocators, and use as a {\em type-name} in a
{\em result-variable-declaration} that also has
a {\em pointer-type-name} (i.e., as a target of a pointer).

The allocators work because the size and alignment of a defined
type are computed as run-time constants initialized when the
program is linked, which is after all extensions of the type have been
declared, and in addition newly allocated values are
always zeroed.  Use as a pointer target is does not require
declaration until the pointer is used to access a value.

Some uses of a {\em defined-type-name} require the type's
size and alignment be known at the compile-time point of usage.
One of these is use
as the {\em type-name} in a {\em result-variable-declaration}
that has \underline{no} {\em pointer-type-name}.

An example is:

\begin{indpar}\begin{verbatim}
ap *READ-WRITE* @X =@ local
                    // Legal, my type need not be declared
ap ro @Y = @X       // Legal, only ap copied.
my type Z = X       // Illegal, my type must be declared.
type my type:
    int I
X.I = 55            // Legal, .I has been declared.

type my type:       // Extension of my type
    int J           // Now X.J == 0
X.J = 66            // Legal, .J has been declared.

my type W:          // my type cannot be extended after this
    W.I = 2         // except by overlays.
    W.J = 10
type my type:       // Legal, does not change size or
    offset 0        // alignment.
    int K1
    int K2
type my type:       // Illegal, increases size.
    int K3
\end{verbatim}\end{indpar}

\subsubsubsection{Pointer Type Declarations}
\label{POINTER-TYPE-DECLARATIONS}

A pointer type has an associated data type specified by
a pointer type declaration.  The syntax is:

\begin{indpar}
\emkey{pointer-type-declaration}\label{POINTER-TYPE-DECLARATION} ::= \\
\hspace*{0.5in}\ttkey{pointer type} {\em defined-pointer-type-name}
	       \TT{is} {\em type-name}
\\[0.5ex]
\emkey{pointer-type-name} ::= see \pagref{POINTER-TYPE-NAME}
\\[0.5ex]
\emkey{type-name} ::= see \pagref{TYPE-NAME}
\end{indpar}
An {\tt *UNCHECKED*} conversion is defined from the associated data
type to data with the given pointer type, and an normal conversion
is defined in the other direction.

For example, the following are builtin:

\begin{indpar}\begin{verbatim}
pointer type dp is int
pointer type ap is data for ap
type data for ap:
    dp ro int @base
    int offset

function ap Q$1 T$1 r = *UNCHECKED* ( data for ap dap )
function data for ap r = pointer to data ( ap Q$1 T$1 ptr )
    // These functions just copy the argument value to the
    // result value changing the type of the value.  Here
    // Q$1 is a wild-card that matches any list of qualifier-names,
    // and T$1 is a wild-card that matches any type-name.

// This function enables implicit conversion of `dp ...' to
// `ap ...', where the latter has the constant 0 for a base
// and the dp value for its offset.
// 
function data for ap r = *IMPLIED* *POINTER* *CONVERSION*
        ( int d ):
    dp int @zero = constant int 0
        // constant int 0 allocates an int equal to 0 to read-only
        // memory whose location is constant during execution
    data for ap dap:
        dap.@base  = @zero
        dap.offset = d
    r = dap

// This function enables *UNCHECKED* conversion of `ap ...' to
// `dp ...' where the latter is the sum of the base and offset
// of the ap.
//
function int r = *UNCHECKED* *POINTER* *CONVERSION*
        ( data for ap d ):
    r = d.base + d.offset
\end{verbatim}\end{indpar}

A {\em pointer-type-declaration} `{\tt pointer type $P$ is $D$}'
implicitly declares the functions:
\begin{indpar} \tt
function $P$ Q\$1 T\$1 r = *UNCHECKED* ( $D$ data ) \\
function $D$ r = pointer to data ( $P$ Q\$1 T\$1 ptr )
\end{indpar}
These just copy values changing type.

In order to access the value pointed at by a pointer of a defined
pointer type, one of several approaches may be taken.  The easiest
is to define a function with the prototype:
\begin{indpar} \tt
function data for $BIP$ r = *ACCESS* *POINTER* *CONVERSION* ( $D$ data )
\end{indpar}
which converts the data for $P$ to the data for a builtin pointer type
$BIP$ ({\tt dp}, {\tt ap}, {\tt fp}, {\tt av}, or {\tt fv}).  If this
is defined, whenever a pointer of type $P$ is to be used as
the address of the value pointed at or a member or vector element
of this value,
the pointer of type $P$ is converted to a pointer of type $BIP$.

A second more limited way to provide for use of $P$ is to define
the two functions:
\begin{indpar} \tt
function Q\$1 T\$1 r = *POINTER* *ACCESS* ( $D$ data ) \\
function *POINTER* *ACCESS* ( $D$ data ) = Q\$1 T\$1 r
\end{indpar}

This allows the pointer to be used to read a copy of the
value pointed at, or write the value.  It does not allow
members or elements of the value to be accessed
(members and elements of the copy may be accessed).

An example is:

\begin{indpar}\begin{verbatim}
type file:
    *READ-WRITE* av uns8 name
    . . . . .
av *READ-WRITE* file @files = global file[1000];
ap *READ-WRITE* int @next file = global int

pointer type file descriptor is int
function data for ap r = *ACCESS* *POINTER* *CONVERSION*
        ( int data ):
    ap file @f = @files[data]
    r = pointer to data ( @f )
        // Qualifiers of @f are ignored by pointer to data.

function file descriptor *READ-WRITE* r = allocate file desciptor:
    r = *UNCHECKED* ( next file )
    next file = next file + 1

file descriptor *READ-WRITE* @fd = allocate file descriptor
fd.name = ...
. . . . .
av uns8 n = fd.name
. . . . .
\end{verbatim}\end{indpar}




\subsubsubsection{Inline Function Declarations}
\label{INLINE-FUNCTION-DECLARATIONS}

The syntax of a function declaration is:

\begin{indpar}
\emkey{function-declaration}\label{FUNCTION-DECLARATION}
    ::= \begin{tabular}[t]{l}
        {\em function-prototype} \TT{:} \\
	\TT{~~~~~}{\em statement}\PLUS{}
	\end{tabular}
\\[2ex]
\emkey{function-prototype}\label{FUNCTION-PROTOTYPE} ::= \\
\hspace*{0.25in}
    \begin{tabular}[t]{@{}rl}
        & \ttkey{function}~
          {\em prototype-result-list}~ \TT{=}~
          {\em module-abbreviation}\QMARK{}~
	                {\em prototype-pattern} \\
    $|$ & \ttkey{function}~ {\em module-abbreviation}\QMARK{}~
                           {\em prototype-pattern} \\
    $|$ & \ttkey{function}~ {\em module-abbreviation}\QMARK{}~
                           {\em prototype-pattern}~ \TT{=}~
                           {\em input-variable-list} \\
    \end{tabular}
\\[0.5ex]
\emkey{prototype-result-list}\label{PROTOTYPE-RESULT-LIST} ::= \\
\hspace*{0.5in}
    {\em prototype-result-declaration}
    \{ \TT{,} {\em prototype-result-declaration} \}\STAR{}
\\[0.5ex]
\emkey{prototype-result-declaration} \\
\hspace*{1in}\begin{tabular}[t]{rl}
    ::= & {\em result-variable-declaration} \\
    $|$ & {\em next-variable-declaration} \\
    \end{tabular}
\\[0.5ex]
{\em module-abbreviation} ::= see \pagref{MODULE-ABBREVIATION}
\\[0.5ex]
{\em result-variable-declaration} ::= see \pagref{RESULT-VARIABLE-DECLARATION}
\\[0.5ex]
{\em next-variable-declaration} ::= see \pagref{NEXT-VARIABLE-DECLARATION}
\\[0.5ex]
\emkey{input-variable-list}
    ::= {\em argument-declaration}
                 \{ \TT{,} {\em argument-declaration} \}\STAR{}
\\[0.5ex]
\emkey{argument-declaration}\label{ARGUMENT-DECLARATION} \\
\hspace*{1in}\begin{tabular}[t]{@{}rl@{}}
    ::= & {\em result-variable-declaration}
          \{ \TT{?=} {\em default-value} \}\QMARK{} \\
    $|$ & \TT{bool} {\em variable-name}
          \TT{??} {\em default-value} \\
    $|$ & {\em result-variable-declaration}
          \TT{==} {\em required-value} \\
    \end{tabular}
\\[0.5ex]
\emkey{default-value} ::= {\em expression}
\\[0.5ex]
\emkey{required-value} ::= compile-time {\em expression}
\\[0.5ex]
{\em expression} ::= see \pagref{EXPRESSION}
\\[0.5ex]
\emkey{prototype-pattern}\label{PROTOTYPE-PATTERN}
    \begin{tabular}[t]{rl}
    ::= & {\em first-pattern-term}~ {\em pattern-term}\STAR{} \\
    $|$ & {\em pattern-argument-list}~ {\em pattern-argument-list}\PLUS{}
    \end{tabular}
\\[0.5ex]
\emkey{first-pattern-term} ::= {\em pattern-argument-list}\STAR{}~
				{\em pattern-term}
\\[0.5ex]
\emkey{pattern-term}
    ::= {\em function-term-name}~ {\em pattern-argument-list}\STAR{}
\\[0.5ex]
\emkey{function-term-name} ::= see \pagref{FUNCTION-TERM-NAME}
\\[0.5ex]
\emkey{function-variable-name}\label{FUNCTION-VARIABLE-NAME} ::= \\
\hspace*{1in}
    \begin{tabular}[t]{@{}p{5in}@{}}
    {\em function-term-name} $N$ that appears in a {\em function-prototype}
    of the form `{\tt function $N$ = \ldots}'
    \end{tabular}
\\[0.5ex]
\emkey{pattern-argument-list}\label{PATTERN-ARGUMENT-LIST} \\
\hspace*{1in}
    \begin{tabular}[t]{@{}rl}
    ::= & \TT{(} {\em argument-declaration}~
                 \{ \TT{,} {\em argument-declaration} \}\STAR{} \TT{)} \\
    $|$ & \TT{[} {\em argument-declaration}~
                 \{ \TT{,} {\em argument-declaration} \}\STAR{} \TT{]} \\
    \end{tabular}
\begin{itemize}
\item
A {\em prototype-pattern} or {\em function-call}
must have either a {\em function-term-name}
or at least two {\em argu\-ment-lists}.
\item
A {\em prototype-pattern} {\em function-term-name} must not be
an initial segment of any other {\em function-term-name}
in the same {\em prototype-pattern}.
\item
A {\em function-variable-name} should follow Rule~\ref{VARIABLE-NAME-RULE}
on \pagref{VARIABLE-NAME-RULE}, that is, it
should not begin with a {\em module-abbreviation},
{\em qualifier-name}, {\em pointer-type-name}, or {\em type-name}.
\item
For a {\em prototype-result-declaration} of the form `\TT{next} $v$',
$v$ must be the {\em vari\-able-name} in an {\em argument-declaration}
of the form `\dots{} {\em type-name} $v$', and
any actual argument associated to the {\em argument-declaration}
must be a {\em variable-name} $w$ for which `\TT{next} $w$' is a legal
{\em assignment-statement} {\em next-variable-declaration}.
\item
All the result and argument {\em variable-names}
in a {\em function-prototype} must
be distinct, with an exception for the previous paragraph.
\item
The first {\em argument-declaration} in an {\em input-variable-list}
must not have a {\em default-value}.
\item
In a {\em pattern-argument-list} or {\em input-variable-list}
an {\em argument-declaration} with no {\em de\-fault-value} cannot
follow an {\em argument-declaration} with a {\em default-value}.
\item
A wild-card (\pagref{WILD-CARD}) name of the form {\tt T\$\ldots}
is treated in a {\em function-prototype} as a {\em type-name}.
A wild-card name of the form {\tt P\$\ldots} is treated as a
{\em pointer-type-name}.
A wild-card name of the form {\tt Q\$\ldots} is treated as a
{\em qualifier-name} and \underline{must not} be combined with
other {\em qualifier-names} in the same {\em result-variable-declaration}
or {\em argument-declaration}.

\end{itemize}
\end{indpar}

An example of an inline function declaration and an inline function call is:
\begin{indpar}\begin{verbatim}
function F ( int x ?= 5 ) G ( int y ) H ( int z ?= 7 ) I ( int w ):
    . . . . . . . . . .
F I ( 8 ) G ( 6 )    // Equivalent to F ( 5 ) G ( 6 ) H ( 7 ) I ( 8 )
\end{verbatim}\end{indpar}

The {\em function-term-names} in the declaration are matched to those
in the call, but need not have the same order in the call, except for
the first {\em function-term-name} which must be the same in the
declaration and the call.  Thus the {\em call-terms} of the call
are re-ordered to match the order of the {\em pattern-terms} of the
declaration.  If one of the {\em pattern-terms} is omitted in the
call, but its arguments have {\em default-values}
the {\em pattern-term} with its
{\em default-values} will be inserted into the call
(here {\tt H ( 7 )} is inserted).
Similarly with an {\em argument-list} that is omitted
(here {\tt ( 5 )} is inserted).

An example containing an {\em input-variable-list} is:
\begin{indpar}\begin{verbatim}
function F [ int x ] = int y,  int z ?= 5:
    . . . . . . . . . .
F[10] = 6
\end{verbatim}\end{indpar}
which is treated as if {\tt =} were a {\em function-term-name}
that must be the last such in the call, and the comma separated
values in the call and {\em argument-declarations} in the prototype
were surrounded by parentheses {\tt (~)}.  Note that for an argument
list in the prototype to match an argument list in the call, both
must be surrounded by the same brackets; either both have {\tt (~)}
or both have {\tt [~]}.

Note that {\em quoted-marks} and {\em quoted-separators}
in {\em function-term-names} appear without quotes in {\em call-term-names}
(see \pagref{CALL-TERM-NAME}).  Thus we have the example:
\begin{indpar}\begin{verbatim}
function int z = ( int x ) "@@" ( int y ):
    . . . . . . . . . .
int w = 5 @@ 6
\end{verbatim}\end{indpar}

A {\em pattern-term} with the syntax:
\begin{indpar}
\emkey{boolean-pattern-term}\label{BOOLEAN-PATTERN-TERM} ::= \\
\hspace*{1in} {\em function-term-name} \TT{(}
        \TT{bool} {\em variable-name}
	\TT{??} {\em default-value} \TT{)}
\end{indpar}

triggers special syntax in a call that matches the prototype.
In the call:
\begin{center}
\begin{tabular}{rcl}
{\em function-term-name} & is equivalent to
                         & {\em function-term-name} \tt ( true ) \\
\TT{no} {\em function-term-name} & is equivalent to
                         & {\em function-term-name} \tt ( false ) \\
\TT{not} {\em function-term-name} & is equivalent to
                         & {\em function-term-name} \tt ( false ) \\
omitted {\em function-term-name} & is equivalent to
                         & {\em function-term-name}
			   \TT{(} {\em default-value} \TT{)} \\
\end{tabular}
\end{center}
Thus the example:
\begin{indpar}\begin{verbatim}
function F ( int x ) OPTION ( bool y ?? true )
    . . . . . . . . . .
F ( 5 )            // Equivalent to F ( 5 ) OPTION ( true )
F ( 5 ) OPTION     // Equivalent to F ( 5 ) OPTION ( true )
F ( 5 ) no OPTION  // Equivalent to F ( 5 ) OPTION ( false )
\end{verbatim}\end{indpar}

If a {\em required-value} is given in a prototype, the call must
have an equal compile-time actual argument value
in order for the call to match the prototype.
Note that the argument variable type need not be compile-time,
as compile-time values can be converted to run-time values.
Matches to prototypes with more {\em required-values}
are preferred over matches to prototypes with less {\em required-values}.
Thus the example:
\begin{indpar}\begin{verbatim}
function F ( int x ) G ( int y == 5 ):  // First F declaration
    . . . . . . . . . .
function F ( int x ) G ( int y ?= 5 ):  // Second F declaration
    . . . . . . . . . .
int z = 5
F ( 8 )            // Matches only second F declaration
F ( 8 ) G ( 5 )    // Matches preferred first F declaration
F ( 8 ) G ( 6 )    // Matches only second F declaration
F ( 8 ) G ( z )    // Matches only second F declaration
                   // (z is not compile-time).
\end{verbatim}\end{indpar}

\subsubsubsection{Inline Call-Prototype Matching}
\label{INLINE-CALL-PROTOTYPE-MATCHING}

Function calls are matched to function prototypes.  It is
a compile error if a function call fails to match any
prototype.  If the call matches more than one prototype, the
matches are ranked according to the number of {\em required-values}
in each prototype, and if there is just one with the maximum
number of {\em required-values}, that match is used; otherwise
it is a compile error.

Call-prototype matching is done as follows:

\begin{enumerate}
\item The {\em function-term-names} in the prototype are matched to
{\em call-term-names} in the call.  To match, the names must be identical,
except that quotes in prototype
{\em quoted-marks} and {\em quoted-separators} are removed in the call
(thus prototype {\tt "+"} matches call {\tt +}).

The match is made by scanning the call from left-to-right
while identifying sequences of lexemes that match
{\em function-term-names} in the prototype.  After identifying
a name, the scan skips to just after the name.  If several
names match at the same position, the longest is choosen.
The scan may match a single prototype name to several points in the
call, but if this happens, the call-prototype match fails.
If the first prototype name fails to match the first call name,
the call-prototype match fails, but otherwise names may be matched
in any order.

\item
The {\em function-term-names} found in the call are used to determine
the extent of {\em call-terms} in the call.  For starters, each
{\em call-term} consists of its {\em call-term-name} and everything
following up to the next {\em call-term}.  If the prototype begins
with {\em pattern-argument-lists}, the situtation is treated as
if both prototype and call began with identical virtual
{\em term-names}.

Then if a {\em call-term-name} is matched to a {\em boolean-pattern-term}
{\em function-term-name} and if its {\em call-term} has no
{\em call-argument-lists}, then if the preceding {\em call-term}
ends in `\TT{no}' or `\TT{not}', this last is removed from the
preceeding {\em call-term} and `{\tt (false)}' is appended to the
current {\em call-term}, and otherwise `{\tt (true)}' is appended to the
current {\em call-term}.

A {\em call-term}
must match its corresponding prototype {\em pattern-term} according
the rules that follow.  Failure of any call-prototype term match
causes the prototype-call match to fail.

\item For a {\em call-term} to match its corresponding {\em pattern-term},
both must have the same number of {\em argument-lists}, the same
brackets (either {\tt (~)} or {\tt [~]}) for corresponding
{\em argument-lists}, and the same number of
arguments in corresponding {\em argument-lists}, \underline{after}
the {\em call-term} has been \key{adjusted}.  The following are
permited adjustments.

For every {\em pattern-term} that has no corresponding {\em call-term}
(because its {\em function-term-name} was not found in the call),
a {\em call-term} consisting of just the {\em pattern-term}'s
{\em function-term-name} is appended to the {\em function-call}.
After this the {\em call-terms} are re-ordered so their order
matches that of their associated {\em pattern-terms}.

If in a left-to-right scan of a {\em call-term},
a {\em call-argument-list} with {\tt (~)} is expected but no {\tt (}
is found, and instead a {\em constant} or {\em reference-expression}
is found, {\tt (~)} parentheses are placed around the
{\em constant} or {\em reference-expression}.  This cannot be
done twice in succession.

If in a left-to-right scan of a {\em call-term},
a {\em call-argument-list} with {\tt (~)} is expected but no {\tt (}
is found, and instead a {\em call-argument-list} with {\tt [~]}
or the end of the {\em call-term} is found,
the empty list {\tt ()} is inserted.

Note that {\em argument-lists} with {\tt [~]} brackets cannot
be omitted or have their {\tt [~]} brackets omitted.

If a {\em call-argument-list} is shorter than the
corresponding {\em pattern-argument-list}, {\em default-values}
in the {\em pattern-argument-list} are inserted in to corresponding
positions in the {\em call-argument-list}.

At this point the {\em pattern-argument-lists} in the prototype
must match in order all the {\em call-argument-lists} in the call, both in
type of brackets (either `{\tt (~)}' or `{\tt [~]}') and in number
of arguments, else the call-prototype match fails.

\item If all the above is successful, then {\em actual-arguments}
in the call are matched to corresponding {\em argument-declarations}
in the prototype according to the rules that follow.
Failure of any of these matches causes
the call-prototype match to fail.

\item If an {\em argument-declaration} has a {\em required-value},
its matching {\em actual-argument} must be compile-time with
a value equal to the {\em required-value}, else the call-prototype
match fails.

\item If the {\em function-call} is an
{\em assignment-right-side}\pagnote{ASSIGNMENT-RIGHT-SIDE},
the number of {\em assignment-results} on the
{\em assignment-left-side} must not be greater than the number of
{\em prototype-result-valuable-declarations}, else the call-prototype
match fails.

The {\em assignment-results} are matched to the
{\em prototype-result-declarations} going from left to right
according to the rules that follow.
Any unmatched {\em prototype-result-declarations}
are ignored.  

\item If an {\em argument-result} $AR$ matches a
{\em prototype-result-declaration} $RD$, the statement `$AR$ = $RD$'
must successfully compile
when the {\em variable-name} in $RD$ is replaced by a unique virtual
{\em variable-name}, else the call-prototype match fails.

The statements `$AR$ = $RD$' are processed in left to right order of
{\em prototype-result-declarations}.
and any unassigned wild-cards (\pagref{WILD-CARD})
in $RD$ are assigned the corresponding
values in $AR$ (e.g., `{\tt int x = T\$1 y}' would assign `{\tt int}'
to {\tt T\$1} if {\tt T\$1} were unassigned).

\item If an {\em argument-declaration} $AD$ matches an {\em actual-argument}
$AA$, the statement `$AD$ = $AA$' must successfully compile
when the {\em variable-name} in $AD$ is replaced by a unique virtual
{\em variable-name}, else the call-prototype match fails.

The statements `$AD$ = $AA$' are processed in left to right order
of prototype {\em argument-declarations},
and any unassigned wild-cards (\pagref{WILD-CARD})
in $AD$ are assigned the corresponding
values in $AA$ (e.g., `{\tt T\$1 x = int y}' would assign `{\tt int}'
to {\tt T\$1} if {\tt T\$1} were unassigned).

\item A {\em function-call} that is \underline{not} an
{\em assignment-right-side}\pagnote{ASSIGNMENT-RIGHT-SIDE}
must be compile-time\pagnote{COMPILE-TIME} else the
call-prototype match fails.
The prototype of such a call must
have at least one {\em prototype-result-declaration} (which
will take on the value of the {\em expression} that consists of
just the {\em function-call}), else the
call-prototype match fails.

Compile-time calls are processed inside-out within a {\em statement},
and upon being processed each call is replaced by its value computed
at compile-time.

\end{enumerate}

\subsubsubsection{Out-of-Line Function Declarations}
\label{OUT-OF-LINE-FUNCTION-DECLARATIONS}

An out-of-line function prototype is a limited subset of
an inline function prototype which ensures that there is
a single ordered list of arguments.  To obtain a more
flexible interface, an out-of-line function call should
be embedded in an inline function that pre-processes the
arguments.

The syntax of an out-of-line function declaration is:

\begin{indpar}
\emkey{out-of-line-function-declaration}%
	\label{OUT-OF-LINE-FUNCTION-DECLARATION}
    ::= \begin{tabular}[t]{l}
        {\em out-of-line-function-prototype} \TT{:} \\
	\TT{~~~~~}{\em statement}\PLUS{}
	\end{tabular}
\\[2ex]
\emkey{out-of-line-function-prototype}%
	\label{OUT-OF-LINE-FUNCTION-PROTOTYPE} ::= \\
\hspace*{0.25in} \ttkey{out-of-line function}~
          \{ {\em prototype-result-list}~ \TT{=}~ \}\QMARK{} \\
\hspace*{0.5in}{\em out-of-line-function-name}~
	      {\em pattern-argument-list}\QMARK{}
\\[0.5ex]
\emkey{out-of-line-function-name} ::= \\
\hspace*{0.25in}
    {\em module-abbreviation}\QMARK{} {\em basic-name}
    $|$ {\em external-function-name}
\\[0.5ex]
{\em external-function-name} ::= {\em quoted-string}
\\[0.5ex]
{\em prototype-result-list} ::= see \pagref{PROTOTYPE-RESULT-LIST}
\\[0.5ex]
{\em module-abbreviation} ::= see \pagref{MODULE-ABBREVIATION}
\\[0.5ex]
{\em basic-name} ::= see \pagref{BASIC-NAME}
\\[0.5ex]
{\em pattern-argument-list} ::= see \pagref{PATTERN-ARGUMENT-LIST}

\begin{itemize}
\item
The rules for inline {\em function-declarations} on
\pagref{FUNCTION-DECLARATION} must be followed where applicable.
\item
`{\tt ??}' {\tt bool} defaults are not allowed.
\item
Wild-cards are not allowed.
\end{itemize}
\end{indpar}

An out-of-line function with a {\em quoted-string} as its
{\em out-of-line-function-name} is \key{external}.  Other
out-of-line functions are \key{internal}.

An internal out-of-line function can be called with a normal
{\em function-call} (\pagref{FUNCTION-CALL}).  An external
out-of-line function must be called with an:
\begin{indpar}
\emkey{out-of-line-function-call}%
	\label{OUT-OF-LINE-FUNCTION-CALL} ::= \\
\hspace*{0.25in}
	\ttkey{call} {\em external-function-name}
	     {\em call-argument-list}\QMARK{}
\\[0.5ex]
{\em call-argument-list} ::= see \pagref{CALL-ARGUMENT-LIST}
\end{indpar}



\subsubsubsection{Module Declarations}
\label{MODULE-DECLARATIONS}

A \key{module} is a file whose first statement is a {\em module-declaration}:

\begin{indpar}
\emkey{module-declaration}\label{MODULE-DECLARATION}
    \begin{tabular}[t]{rl}
    ::= & {\em simple-module-declaration} \\
    $|$ & {\em simple-module-declaration}\TT{:} \\
	& \TT{~~~~}{\em import-clause}\STAR{} \\
    \end{tabular} \\
\emkey{simple-module-declaration} ::= \TT{module} {\em module-name}
        \TT{as} {\em module-abbreviation} \\
\emkey{module-name}\label{MODULE-NAME} ::= {\em quoted-string} \\
\emkey{module-abbreviation}\label{MODULE-ABBREVIATION}
	::= \TT{word} not containing any `\TT{.}'s \\
\emkey{import-clause}\label{IMPORT-CLAUSE}
    ::= \ttkey{import} {\em module-name} \TT{as} {\em module-abbreviation}

\begin{itemize}

\item
A {\em module-declaration} may only appear as the first statement
of a module file.

\item
In a {\em module-declaration} all {\em module-abbreviations} must be
distinct, and all {\em module-names} must be distinct.
\end{itemize}
\end{indpar}

A {\em module-name} is a POSIX file name.
The {\em module-name} of a {\em module-declaration} must match the
name of the file containing the
{\em module-declaration}, relative to one of several directories
specified separately to the compiler.

The {\em module-abbreviation} associated with a {\em module-name}
may differ in different files.  Specifically, the {\em module-abbreviation}
for a module used in the module's own module file need not be the same
as the {\em module-abbreviations} used for the module in files
that import the module.

The module \TT{"standard"}\index{standard@\TT{"standard"}} with
module abbreviation \ttkey{std} is builtin and contains the builtin types and
functions.  The {\em import-clause}
\begin{center}
{\tt import }\TT{"standard"}{\tt{} as \ttkey{std}}
\end{center}
is implied in every {\em module-declaration} and
{\em body-declaration}.

A \key{body} is a file whose first statement is a {\em body-declaration}:

\begin{indpar}
\emkey{body-declaration}\label{BODY-DECLARATION} ::=
    \begin{tabular}[t]{l}
    \TT{body }{\em body-name}\TT{ of }{\em module-name}\TT{:} \\
    \TT{~~~~}{\em body-clause}\STAR{} \\
    \end{tabular}
\\[0.5ex]
\emkey{body-name} ::= {\em quoted-string}
\\[0.5ex]
{\em module-name} ::= see \pagref{MODULE-NAME}
\\[0.5ex]
\emkey{body-clause} ::= {\em import-clause} $|$ {\em after-clause}
\\[0.5ex]
{\em import-clause} ::= see \pagref{IMPORT-CLAUSE}
\\[0.5ex]
\emkey{after-clause} ::= \ttkey{initialize after }{\em body-name}

\begin{itemize}

\item
A {\em body-declaration} may only appear as the first statement
of a body file.

\item
In a {\em body-declaration} the {\em module-abbreviations} of imported
modules must be distinct and must be different from the
{\em module-abbreviation} used by the body's module,
and all {\em module-names} and {\em body-names} must be distinct.
\end{itemize}

\end{indpar}

A {\em body-name} is a POSIX file name.
The {\em body-name} of a {\em body-declaration} must match the
name of the file containing the
{\em body-declaration}, relative to one of several directories
specified separately to the compiler.

A \key{body} is an extension of the module named in the first
line of the {\em body-declaration}.

A body implicitly imports the module it extends.  Within the
body that module has the same {\em module-abbreviation} that it
had in the module's own file.  The other modules imported in the
module's own file are \underline{not} implicitly imported
to the body.  The body must import whatever other modules it uses
explicitly.

The {\em after-clauses} name other bodies that extend the same
module, and determine the order in which bodies are initialized:
see \itemref{PROGRAM-INITIALIZATION}.

The conceptual directed graph whose nodes are modules and bodies
and whose arrows connect each module or body to the modules it imports
and for a body to the bodies it is `{\tt initialized after}'
is called
the `\key{initialization graph}'\label{INITIALIZATION-GRAPH}
and \underline{must be acyclic}.

\subsubsection{Allocation Statements}
\label{ALLOCATION-STATEMENTS}

The syntax of {\em allocation-statements} is:

\begin{indpar}
\emkey{allocation-statement} ::= \\
\hspace*{0.5in}\begin{tabular}[t]{rl}
        & {\em assignment-left-side}~
	  \ttkey{=@}~ {\em function-call} \\
    $|$ & {\em assignment-left-side}~
	  \{ \ttkey{=@}~ {\em function-call} \}\QMARK{}
	\TT{:} \\
	& \TT{~~~~}{\em statement}\STAR{} \\
	\end{tabular} \\
\\[0.5ex]
\emkey{function-call} ::= see \pagref{FUNCTION-CALL} \\
\emkey{statement} ::= see \pagref{STATEMENT} \\
\\[1ex]
where
\begin{itemize}
\item The {\em assignment-left-side} may \underline{not} contain
{\em reference-expressions} or {\em deferred-results}.
\item If there is a {\em function-call}, all the {\em assignment-left-side}
variables must have pointer types.
\end{itemize}
\end{indpar}

The statement allocates memory for the left side variables and
zeros that memory.  Then any {\em statements} are executed with
the allocated memory being write-only.

More specifically, if there is no {\em function-call}, the
{\em assignment-left-side} variables are allocated to the
current function frame and zeroed, and then the {\em statements}
are executed with these variables being treated as {\tt *WRITE-ONLY*}.
After the {\em allocation-assignment-statement} if finished, these
variables will be treated as {\tt co} (constant, never changed).

If there is a {\em function-call}, the {\em function-call} will be
executed with a pre-pended argument list separately for each
{\em assignment-left-side} variable.  The variable must have pointer
type.  The {\em prototype-pattern} of the called function's prototype
(see \pagref{PROTOTYPE-PATTERN}) must begin with: \\
\centerline{\tt ( uns length, uns alignment )} \\
and when called a {\tt ()} {\em call-argument-list} is pre-pended to
the {\em function-call} giving two {\tt uns} arguments: first the
number of bytes to be allocated, and second the alignment required
of the first byte.  The called function must allocate a block of
memory with the required number of bytes and alignment, zero that
block, and return a pointer to the block.  The prototype must have
exactly one result variable which is a pointer.

If there is both a {\em function-call} and {\em statements},
the {\em statements} are executed after all {\em assigment-left-side}
variables have been set.  Within the block containing
these {\em statements},
the target qualifiers of these pointer variables are effectively changed to
{\tt *WRITE-ONLY*}.

As a general rule, allocator functions have a {\tt []} argument list
with a single argument giving a vector size {\tt N} that defaults to {\tt 1}.
The allocator then allocates not a single block of the given length
and alignment, but instead a vector of {\tt N} such blocks.  However,
this is by convention and is not a builtin feature of the L-Language.

\section{Compile Time}

Statements and functions, all of whose variables are of {\tt const}
type, execute solely during compilation, and never at run time
(when the compiled program is running).  Constants are all of
type {\tt const}, and cannot exist at run-time, though number
and quoted string {\tt const} values can be converted to run-time values.

\subsection{Compile Time Data}

Compile time data consists of {\tt const} values.
There are 5 types of {\tt const} values: integers, rationals, strings,
special values, and maps.

\key{Integers} are unbounded and can be arbitrarily long.
Any sequence of lexemes that is a
number representative
that evaluates to an integer
can be used to represent an integer {\tt const}.

\key{Rationals} are pairs of unbounded integers, a numerator and a denominator.
The demoninator must be greater than or equal to 2.  The greatest common divisor
of the numerator and denominator must be 1.
Any sequence of lexemes that is a
number representative
that evaluates to a non-integer
can be used to represent a rational {\tt const}, but not all
rational {\tt const} values can be represented by lexeme sequences
that are not {\em expressions}.

\key{Strings} are character strings, represented by quoted string lexemes.

The \key{special values} defined at compile-time are the booleans
\ttkey{TRUE} and \ttkey{FALSE}, and the value \ttkey{UNDEF},
which denotes the `\key{undefined value}'.

\key{Maps} have two parts.  The \key{vector} part maps small
natural numbers in the range $[0,n-1]$ to values, where $n$
is the length of the vector.  The \key{dictionary} part maps
character string \key{labels} to values.

Labels that begin with '{\tt .}' are \key{hidden} and do not
print.  Maps that have no non-hidden labels are called \key{lists}.

Maps are represented by comma separated lists of elements inside
{\tt [~]} brackets, with vector elements being represented by
just their value, and dictionary elements being represented
by the syntax `label {\tt =>} value'.

The parser translates an input text into nested lists.  These
lists have a {\tt .position} hidden label whose value specifies
the position in the input of the text that produced the list.
Some of the lists have an {\tt .operator} hidden label whose
value specifies the operator in the list, which has usually
been moved to the beginning of the list.  Logical lines,
indented paragraphs, and parenthesized subexpressions
are lists with {\tt .initiator} and {\tt .terminator}
hidden attributes.

\subsubsection{Maps}
\label{MAPS}

A \key{map} is a compile time {\tt const} value that, unlike
numbers and strings, is mutable.
A {\em map-expression} computes a map value:

\begin{indpar}
\emkey{map-expression} ::=
    \TT{[]} $|$
    \TT{[} {\em map-element} \{ \TT{,} {\em map-element} \}\STAR{} \TT{]}
\\[0.5ex]
\emkey{map-element} ::=
        {\em expression} $|$ {\em data-label} \TT{=>} {\em expression}
	                 $|$ \TT{[} {\em map-index} \TT{]}
			 	\TT{=>} {\em expression}
\\[0.5ex]
{\em data-label} ::= see \pagref{DATA-LABEL}
\\[0.5ex]
\emkey{map-index} ::=
    {\em expression} evaluating to a {\tt const} integer or string
\end{indpar}


Elements of a map can be accessed by {\em map-element-expressions}, which
are also {\em reference-expressions} (see \pagref{REFERENCE-EXPRESSIONS}):

\begin{indpar}
\emkey{map-element-expression}
    \begin{tabular}[t]{rl}
    ::= & {\em map-base} \TT{[} {\em map-index} \TT{]} \\
    ::= & {\em map-base} {\em map-member} \\
    \end{tabular}
\\[0.5ex]
\emkey{map-base} ::= {\em expression} evaluating to a map
\\[0.5ex]
\emkey{map-member} :::= \TT{.} {\em label} ~~~~[note :::= and not ::=]
\end{indpar}

A map is mutable in that its elements may be written as well as read.
Note that hidden labels turn into {\em map-members} beginning with
two dots: the {\tt .position} of {\tt x} is {\tt x..position}.

The {\em label} of a map element, when represented as a string, can
be used in a {\em map-index} to designate the element.  That is,
given a map {\tt x} and a label {\tt y}, {\tt x.y} and {\tt x["y"]}
are equivalent.

There is a virtual hidden element of a map, \ttdkey{length}, which is
the value of the largest integer index of any map element,
plus 1.  So {\tt ["A", "B", "C"]..length == 3}.  If a map has no
integer indices, its {\tt .length} is 0.

Values may be stored into map elements as well as read from the
elements.  Storing into a non-existant element creates the element.

Storing {\tt UNDEF} in an element deletes the element, and reading
from a non-existant element returns {\tt UNDEF} without any other
indication of there being an error.  Storing an integer into the
{\tt .length} of a map deletes all elements with integer indices
greater than or equal to the integer stored (but this may make
the map {\tt .length} less than instead of equal to the integer stored).

A simple assignment {\tt x = y} where {\tt y} is a map makes
a copy of the map in {\tt x}, so that {\tt x} and {\tt y} hold
\underline{different} maps (but they are {\tt ==}).  These two different maps
can be alterred independently.

A simple assignment {\tt @ x = @ y} where {\tt y} is a map makes
{\tt x} and {\tt y} both designate the \underline{same} map.
Altering the map designated by {\tt x} will alter the map
designated by {\tt y} in the same way.

A simple assignment such as {\tt @ x = @ y[z]} makes {\tt x} and
{\tt y[z]} designate the \underline{same} element of the map {\tt y},
so that storing into {\tt x} is the same as storing into
{\tt y[z]}.

Locations can be compared using {\tt ==} and {\tt @}.  So if {\tt y}
is a map, {\tt @ x == @ y[z]} can be used to test whether {\tt x}
and {\tt y[z]} both reference the same map element.

There are no {\tt const} pointer values, so {\tt x = @ y[z]} is not
legal when {\tt y} is a map.

\subsection{Operators and Builtin Functions}

\subsection{Control Structures}

\subsubsection{Function Definitions}

\section{Run Time}

\subsection{Data Types}

\subsubsection{Memory Organization}

\subsubsection{Numeric Data Types}

\subsubsection{Builtin Pointer Data Types}

\subsubsection{Defined Data Types}

\subsection{The Symbol Table}

\subsection{Assignment Statements}

\subsubsection{Operators and Builtin Functions}

\subsection{Control Structures}

\subsubsection{Inline Functions}

\subsubsection{Out-of-Line Functions}




\printindex

\end{document}
